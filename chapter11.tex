\chapter{ORGANIZING A MATH FIELD TRIP}
\section*{INTRODUCTION}
This module presents a guide on how to organize a field trip for Mathematics class. This
encourages the teacher to have an excursion as an alternative activity to promote learning in
Mathematics. This also emphasizes that experience is the best education thus, when students actually
apply their knowledge in a setting outside of school, these experiences strengthen their learning
experience.
\section*{OBJECTIVES}
After completing this module, you should be able to:
\begin{enumerate}
\item Learn different guidelines in organizing a field trip for Mathematics class; and
\item Appreciate outdoor activities like field trips to promote learning in Mathematics.
\end{enumerate}
\section*{DISCUSSION}
\subsection*{Importance of Field Trips}
Field trips can be a wonderful avenue for students to have a firsthand experience on the areas
of the subject matter. It is an opportunity for the students to explore the rich resources of the local
community. Through such, students’ knowledge and understanding of the subject will increase while
exposing them to reality. Moreover, certain values such as sense of responsibility, independence, team
work and cooperation among students will be developed.
\subsection*{Field Trip for Mathematics Class}
Math field trips can cover a whole range of venues and activities. Here are a few sample math
field trip venues to think about:
\begin{itemize}
\item garden to see the growth patterns in plants (i.e. Fibonacci, etc.);
\item  farm to calculate the farm's yield;
\item  forest area to record the ages of trees;
\item  factory to explain how math affects each level of the process and how math is used in the design
and manufacture of the products (This could also be a venue to appreciate linear
programming.);
\item  airport where students can plot plane takeoff/landing trajectories;
\item  tall landmark such as a nearby building for the students to calculate the height of the building
using trigonometry;
\item 
train station to help students understand the passage of time, estimating time, calculating the
difference between arriving and departing trains, and managing time to arrive on time for a
departing train;
\item 
sport event to record statistics at the event (For example, the number of assists and three-point
shots in a basketball game. Averages can be calculated as part of classroom activities on the
next school day.);
\item  circus or theme park to look into the essence of symmetry and rotations;
\item  local park to apply the concept of measurements (perimeter and area);
\item  bank to give students exposure to money management as they learn about the importance of
savings and checking accounts; and
\item 
historical building to look into Golden ratio, tessellations and isometries.
\end{itemize}
\subsection*{Tips in Organizing a Field Trip}
The success of the field trip lies on a good plan. Careful attention should be given to trip
selection, pre-visit preparation, the trip itself, and evaluation. When considering a field trip, teachers are
advised to first consult with their administrator regarding existing school board policies and follow those
recommended procedures.

Herein is a guide adopted from \parencite{campsilos}.

\subsubsection*{Trip Selection}
\begin{enumerate}
\item Identify the rationale, objectives and plan of evaluation for the field trip.
\item Select the site to be visited. Contact the educational coordinator for the site and arrange the
date and time. Obtain the pre-trip information package if one is available. Record addresses,
directions, names of contact persons and their phone numbers, e-mail addresses, etc.
\item Conduct a pre-visit to familiarize yourself with the major features of the venue. Take
photographs to share with students prior to the visit. Explore the place to get ideas for activities.
\end{enumerate}
\subsubsection*{Logistics Planning}
\begin{enumerate}
\item File the activity permit.
\item Make transportation arrangements. File request for use of school bus, if possible and applicable.
\item Make arrangements for meal or packed lunch, if needed.
\item Outline the schedule for the day.
\item Make a seat plan for the bus.
\item Arrange for special equipment and supplies, such as video camera and digital camera.
\item Prepare name tags for students and chaperones.
\item Collect money for admission fees.
\item Compose parent permission letter including
  \begin{itemize}
	\item Date and location of the field trip and transportation arrangements
	\item Educational purpose of the field trip
	\item Provision for students with special needs
	\item Cost of the trip
	\item List of things that the students need to bring
	\item Lunch arrangements
	\item Money needed
	\item Attire for the trip
	\item Trip schedule
	\item Whether a child will need prescribed medication administered
	\item Reply slip with parent's name and signature
	\end{itemize}
\item Send a letter to parents or include in the class newsletter a request for volunteers as
chaperones.
\item Communicate assigned duties/responsibilities, review field trip objectives, and list activities and
schedule.
\item Provide alternative activities for students who will not be going on the trip.
\item Submit a list of students who will be attending the field trip to other teachers if their classes will
be affected.
\item Collect the money for the trip and deposit it to the school's account. If required, send the
advanced fee to the field trip site.
\item Create a list of all student names and home phone numbers for use in an emergency.
\end{enumerate}
\subsubsection*{Preparing Students Before the Trip}
\begin{enumerate}
\item Discuss the purpose of the field trip and how it relates to the current unit of study.
\item Show photographs of the field trip site.
\item Explore the website of the location you will be visiting.
\item As a class, brainstorm a set of standards of conduct for the trip and discuss suggested spending
money, lunch plans, appropriate clothing to wear for the trip including gear for rainy weather.
\item Assign students "specialists" roles in one aspect of the topic that they will be studying during the
field trip. Students could be grouped in different subject areas related to the field trip topic to
research.
\item Discuss with students how to ask good questions and brainstorm a list of open-ended
observation questions to gather information during the visit. Have the students record the
questions on their field trip journals.
\item Overview the field trip schedule.
\end{enumerate}
\subsubsection*{Final Planning}
Check all permission slips the day before the field trip.
\subsubsection*{Conducting the Trip}
On the day of the trip:
\begin{itemize}
\item Pass out name tags.
\item Divide class into small groups and assign chaperones to groups.
\item Assign each student a partner/buddy.
\item Secure a class list and student emergency forms in a folder.
\item Bring an emergency kit.
\item Take inventory of food, specific equipment, and other supplies pertinent to the trip.
\end{itemize}

\subsubsection*{Activities During the Trip}
Plan activities that allow students to work alone, in pairs or in small groups. Activities might include:
\begin{itemize}
\item Adventure game "Math Trail..."
\item Mystery with clues provided
\item Sketch pages with partial drawings of objects found in the venue for students to complete the
   drawings based on their observations
\item Hand drawn postcards to write near the end of the tour that will summarize the field trip visit
\end{itemize}
Provide time for students to observe, ask questions, and record key words, ideas and phrases as journal
entries in their field book after exploring the place.

Ask follow-up questions as students make observations.
\begin{itemize}
\item How did they apply the concept of mathematics?
\item Can you observe any pattern?
\item What particular concept in mathematics is being illustrated?
\item How are these two objects different from one another?
\item In what ways do these two objects relate to one another?
\end{itemize}
Schedule a particular segment of the field trip for a scavenger hunt where students look for particular
objects and record them in their field book or on an observation sheet.

Provide time for students to work in their field book writing questions, describing favorite displays or
making sketches of artifacts, structures, scenery, etc. If they cannot complete their sketches, encourage
them to label them for future completion as to color, detail, etc.

Provide time for students to use (tape recorder, camcorder, digital camera) for recording important
resources viewed/heard.
\subsubsection*{Post-Field Trip Activities}
Just as quality pre-planning is essential to the success of a field trip, planning for appropriate follow-up
activities will facilitate student learning and multiply the value of hands-on experiences outside the
classroom. The following activities provide a general guide when planning for post-field trip classroom
experiences.

\begin{itemize}
\item Provide time for students to share general observations and reactions to field trip experiences.
\item Share specific assignments students completed while on the field trip.
\item Create a classroom bulletin board displaying materials developed or collected while on the field
   trip.
\item Develop a classroom exhibit that replicates and extends displays students observed on the field
   trip.
\item Link field trip activities to multiple curricular areas. For example, students can develop
   vocabulary lists based on field trip observations; record field trip observations in a classroom
  journal; construct math problems related to actual field trip budget planning; etc.
\item Share and evaluate student assignments/activities from their field books.
\item Have the class compose and send thank-you letters to the field trip site host, chaperones, school
   administrators and other persons that supported the field trip. Include favorite objects or
  special information learned during the field trip.
\item Create a short news report about what happened on the field trip. Publicize the trip via an
   article in your local newspaper, school bulletin board, trip presentation for parent's night, or
  class web page.
\end{itemize}

\subsubsection*{Evaluating the Trip}
Complete a "Teacher Journal" regarding the field trip. This will provide a good reference for future field
trips.
\begin{itemize}
\item What was of unique educational value in this field trip?
\item Did the students meet the objectives/expectations?
\item Was there adequate time?
\item Was there adequate staff and adult supervision?
\item What might be done differently to make this an even better experience in the future?
\item What special points should be emphasized next time?
\item What special problems should be addressed in the future?
\item What would improve a visit to this site in the future?
\end{itemize}
Share the evaluation with the students, volunteers, hosts from the field trip site, and school
administrators.

\section*{SUGGESTED ACTIVITY}
\subsection*{OBJECTIVES}
\begin{enumerate}
\item Apply the concepts of the mean, median, and mode in real life situations.
\item Create/design an activity that requires the use of the concept of the mean, median, and mode.
\end{enumerate}
\subsection*{SUGGESTED TIME ALLOTMENT}
1 day
\subsection*{MATERIALS}
\begin{itemize}
\item meter stick
\item tape measure
\item writing materials
\end{itemize}
\subsection*{PROCEDURES}
The class will be divided into 3 groups. Each group will have a designated area or location where they
will collect their data.
\begin{myenumerate}
\item From the designated area (preferably a park or an area where there are variety of plants),
classify each tree by their names. Count the number of each class of tree and record the result
as the table below:
\begin{center}
\begin{tabular}{|>{\centering\arraybackslash}p{0.25\linewidth}|>{\centering\arraybackslash}p{0.25\linewidth}|}
\hline
Name of Tree & Number of Trees\\
\hline
 & \\ \hline
 & \\ \hline
 & \\ \hline
 & \\ \hline
 & \\ \hline
\end{tabular}
\end{center}

Based on the data, what is the most abundant tree in that location? We call this the mode.
What can you infer based on the data?

\item From the same designated area, look for the ornamental plants (preferably 30 - 50 ornamental
plants). Measure the height of the ornamental plants in centimeters using the meter stick.
Arrange the heights from smallest to tallest or from tallest to highest. Record the result as the
table below.

\begin{center}
\begin{tabular}{|>{\centering\arraybackslash}p{0.25\linewidth}|>{\centering\arraybackslash}p{0.25\linewidth}|}
\hline
Ornamental Plant & Height in cm\\
\hline
1 & \\ \hline
2 & \\ \hline
3 & \\ \hline
4 & \\ \hline
5 & \\ \hline
\end{tabular}
\end{center}

Based on the data, what is the middle most height? We call this the median.
What can you infer based on the data?

\item On the same designated area, measure the girth (stem circumference) of the trees in cm using
the tape measure. The number of trees to be measured is preferably 30 - 50 trees.
Record the result as the table below:

\begin{center}
\begin{tabular}{|>{\centering\arraybackslash}p{0.25\linewidth}|>{\centering\arraybackslash}p{0.25\linewidth}|}
\hline
Tree & Girth in cm\\
\hline
A & \\ \hline
B & \\ \hline
C & \\ \hline
D & \\ \hline
E & \\ \hline
\end{tabular}
\end{center}

Based on the data, sum up the all the girths of trees and then divide it by the number of trees
measured. What is the result? We call this the mean.

What can you infer based on the data?

\end{myenumerate}
