\chapter[SOLVING WORD PROBLEMS]{SOLVING WORD PROBLEMS\\ IN GEOMETRY}
\section*{INTRODUCTION}
This module is intended to give an overview on solving word problems in geometry. It presents
the different steps in problem solving as designed by Polya. It also contains sample word problems for
the participants to solve. Assessment questions are also given.
\section*{OBJECTIVES}
After completion of the module, the participants are expected to:
\begin{enumerate}
\item Apply the different steps in problems solving
\item Determine the appropriate strategy to use in solving a word problem.
\item Use variety of problem solving strategies in solving word problems.
\item Solve the word problems accurately.
\item Appreciate the importance of problem solving in a mathematics curriculum.
\item Gain confidence in doing problem solving
\item Create more interesting word problems.
\item Integrate or connect geometry to origami, the art of paper folding.
\end{enumerate}
\section*{DISCUSSION}
The National Council of Teachers of Mathematics (NCTM) gives the following definitions on the
nature of mathematics: mathematics is a study of patterns and relationships, it is a way of thinking, it is
an art, it is a language and it is a tool. NCTM identified five broad goals required to meet the students’
mathematical needs for the 21st century. They are as follows: students should (1) value mathematics,
(2) reason mathematically, (3) communicate mathematics, (4) solve problems and (5) develop
confidence.

Problem solving is the heart of mathematics. It is a process. It is the means by which an
individual uses previously acquired knowledge, skills and understanding to satisfy the demands of an
unfamiliar situation. Students must be exposed to a variety of problems – problems that vary in context,
in level of difficulty and in methods in solving the problem. But what is really a problem or a word
problem specifically? A word problem is a mathematical proposition that a student has never
experienced solving. What are the purposes of word problems? The following are the different
purposes of word problems: (1) to prepare the students for real life, (2) to develop children’s logical and
abstract thinking and mental discipline, and (3) to motivate students. Hence a word problem should
simulate real life situations, events, and places that students can experience firsthand. In this way the
students will be able to see the importance of mathematics in their life.

George Polya, a mathematician, devised the following \Bold{four-step method} in solving word problems:

\begin{enumerate}
\item \textbf{Read and understand the problem}\index{four-step method!Read and understand the problem}

Read and reread the problem. Ask the following questions: What is the situation all
about? What information is given? What are the different assumptions? What information is
missing? What are you being asked to find or to do? You can make a list in this format

Information given (GIVEN): \makebox[2in]{\hrulefill}

Goals of the problem (REQUIRED): \makebox[2in]{\hrulefill}

\item \textbf{Plan how to solve the problem. Try a strategy}\index{four-step method!Try a strategy}\index{four-step method!Plan how to solve the problem}

In developing a plan to solve the problem ask the following questions: Have you ever
worked a similar problem before? Will you estimate or calculate? What strategy (ies) can you
use? Consider the different strategies in solving problem that you know. The following could be
your shopping list on the different problem solving strategies: draw a diagram, make a table,
write an equation, guess and check (trial and error), look for a pattern, solve a simpler problem,
work backward. Word problems can be solved using different strategies.

\item \textbf{Solve the problem/ Carry out the plan.}\index{four-step method!solve the problem}\index{four-step method!carry out the plan}

As you implement the strategy that you choose, ask the following questions: Did you
calculate correctly? What is the solution? Did you interpret correctly? Did you answer the
question?

\item \textbf{Look back}\index{four-step method!look back}

Most students forget this last part especially if they are solving a problem set or many
problems. The most important question that should be asked here is: “Is the calculated answer
reasonable?”
\end{enumerate}
However there is also a danger in over-using Polya's four steps in problem solving. Students
may think that problem solving is one directional that follows the given steps above as if it's the
prescribed recipe. Emphasize to students that with challenging problems, the actual problem solving
becomes a process whereby the solver keeps a mental "check" of the progress, and corrects himself if
progress is not made. He may go one route, notice it won't work, go backwards a bit, and then, take
another route. In other words, devising plans and carrying them out can occur somewhat
simultaneously, and we can go back and forth. Lastly, not all problems can be solved using the steps
above.

\subsection*{Problems Involving Perimeters}
\begin{example}
\Item The perimeter of a square is 28 cm. Find the length of each side.

\Solution

\textit{Geometry Concept:} A square is a quadrilateral (4-sided polygon) with four congruent sides. Its four
interior angles are all right angles. The formula for the perimeter ($P$) of a square is $P = 4s$ and its area ($A$) is $A = s^2$ where $s$ is the side of the square.

\begin{tikzpicture}
\draw (0,0) -- (1,0) -- (1,1) -- (0,1) -- cycle;
\node [below] at (0.5,0) {$s$};
\node [right] at (1,0.5) {$s$};
\node [above] at (0.5,1) {$s$};
\node [left] at (0,0.5) {$s$};
\end{tikzpicture}
\begin{align*}
P&=4s\\
28&=4s\\
\therefore s&=7
\end{align*}
Therefore, the square has side of length 7 cm.
\end{example}

\begin{example}
\item The length of a rectangular garden is 3 meters more than its width. If the perimeter of the garden is
50 meters, what is the area of the rectangle?

\Solution

\textit{Geometry Concept}

A rectangle is a quadrilateral (4-sided polygon) with two pairs of congruent
and parallel sides. Its four interior angles are all right angles.
The formula for the perimeter ($P$) of a rectangle is $P = 2L + 2W$ where $L$ is
the length and $W$ is the width of the rectangle.

\begin{tikzpicture}[xscale=2]
\draw (0,0) -- (1,0) -- (1,1) -- (0,1) -- cycle;
\node [below] at (0.5,0) {$L$};
\node [right] at (1,0.5) {$W$};
\node [above] at (0.5,1) {$L$};
\node [left] at (0,0.5) {$W$};
\end{tikzpicture}

\textit{Representation}
\begin{align*}
W&=\text{width of the rectangle}\\
L&=W+3
\end{align*}

\textit{Equation:}
\begin{align*}
50 &= 2(W + 3) + 2W && \text{Using the formula for perimeter.}\\
50 &= 2W + 6 + 2W \\
44 &= 4W \\
W &= 11 \\
L &= 11 + 3 = 14 \\
A &= (14)(11) = 154 && \text{Using the formula for area.}
\end{align*}

Therefore, the rectangle has an area of 154 $\cm^2$.
\end{example}

\begin{example}
\Item The longest side of a triangle is twice as long as the shortest side and is also 2 cm longer than the third side. If the perimeter of the triangle is 33 cm, what is the length of each side?

\Solution

\textit{Geometry Concept:}

A triangle is a 3-sided polygon. The formula for the perimeter ($P$) of a triangle is
$P = a + b + c$ where $a$, $b$ and $c$ are the sides of the triangle.

\begin{tikzpicture}
\draw (0,0) -- (2,0) -- (0.5,1.5) -- cycle;
\node [below] at (1,0) {$c$};
\node [right] at (1.25,0.75) {$b$};
\node [left] at (0.25,0.75) {$a$};
\end{tikzpicture}

\textit{Representation}

\begin{align*}
a&=x && \text{shortest side of the triangle}\\
b&=2x && \text{longest side}\\
c&=2x-2 && \text{third side}
\end{align*}

\textit{Equation}

\begin{align*}
33 &= x + 2x + 2x - 2 && \text{(Using the formula for the perimeter.)}\\
35 &= 5x\\
x &=7\\
a&=x=7\\
b&= 2x = 14\\
c &= 2x - 2 = 12
\end{align*}
Therefore, the sides of the triangle are 7 cm, 12 cm and 14 cm.
\end{example}

\begin{example}
\Item What is the measure of the base of an isosceles triangle if its perimeter is 50 cm and the length of one of its congruent sides exceeds the length of the base by 10 cm?

\begin{tikzpicture}[yscale=2]
\draw (0,0) -- (1,0) -- (0.5,1) -- cycle;
\node [below] at (0.5,0) {base};
\node [right] at (0.75,0.5) {leg};
\node [left] at (0.25,0.5) {leg};
\end{tikzpicture}

\Solution

\textit{Geometry Concept}

An isosceles triangle is a triangle with two congruent sides. The congruent sides are
called legs while the non-congruent side is called the base.

\textit{Representation}

\begin{align*}
x&=\text{length of the base of the triangle}\\
	x+10&=\text{length of the leg of the triangle}
\end{align*}

\textit{Equation}

\begin{align*}
50 &= x + 2(x + 10) && \text{(Using the formula for perimeter of triangles.)}\\
50 &= x + 2x + 20\\
30 &= 3x\\
x &= 10
\end{align*}
Therefore, the base of the isosceles triangle is 10 cm.
\end{example}

\subsection*{PROBLEMS INVOLVING ANGLES OF POLYGONS}
\begin{example}
\Item The first angle of a triangle is triple that of the smallest angle. The second angle is 5 degrees more than the first angle. What is the measure of the smallest angle?

\Solution

\textit{Geometry Concept}

A triangle has three interior angles and their measures sum up to 180 degrees.

\textit{Representation}
\begin{align*}
x &= \text{smallest angle of the triangle}\\
3x &= \text{first angle of the triangle}\\
3x + 5 &= \text{second angle of the triangle}
\end{align*}
\textit{Equation}
\begin{align*}
x + 3x + 3x + 5 &= 180 && \text{(Sum of the angles of a triangle is $180^{\circ}$.)}\\
7x &= 175\\
x &= 25
\end{align*}
Therefore, the measure of the smallest angle of the triangle is 25 degrees.
\end{example}

\begin{example}
\Item One of the base angles of an isosceles triangle measures 15 degrees less than its vertex angle. Find the
measure of its vertex angle.

\textit{Geometry Concept}

In an isosceles triangle, it has two congruent angles opposite the congruent sides. The angle opposite
the base is called the vertex angle.

\Solution

\textit{Representation}

\begin{align*}
x&=\text{vertex angle}\\
x-15 &= \text{base angle}
\end{align*}

\textit{Equation}

\begin{align*}
2(x - 15) + x &= 180 && \text{(Sum of the angles of a triangle is $180^{\circ}$.)}\\
2x - 30 + x &= 180\\
3x &= 210\\
x &= 70
\end{align*}
Therefore, the measure of the vertex angle of the triangle is 70 degrees.
\end{example}

\begin{example}
\Item Given the parallelogram on the right, find the value of $x$.

\begin{tikzpicture}
\node (trap1) [trapezium,trapezium left angle=81,trapezium right angle=99,minimum width=3cm, minimum height=1.5cm,draw] {};
\node [below right] at (trap1.north west) {$(6x-15)^{\circ}$};
\node [above right] at ([xshift=-10pt]trap1.south west) {$(4x+5)^{\circ}$};
\end{tikzpicture}

\textit{Geometry Concept}

A parallelogram is a quadrilateral with two pairs of congruent and parallel sides. Its two pairs of opposite
angles are congruent and the sum of the measures of any two consecutive angles is 180 degrees.

\Solution

\begin{align*}
(4x + 5) + (6x - 15) &= 180 && \text{(The given angles are consecutive.)}\\
10x - 10 &= 180\\
10x &= 190\\
x &= 19
\end{align*}
\end{example}

\begin{example}
\Item Given the rhombus, find the measure of the angles of the rhombus.

\begin{tikzpicture}
\node (trap1) [trapezium,trapezium left angle=45,trapezium right angle=135,minimum width=3cm, minimum height=1.5cm,draw] {};
\node [below right] at ([xshift=-15pt]trap1.north east) {$(6x-15)^{\circ}$};
\node [above left] at ([xshift=15pt]trap1.south west) {$(4x+5)^{\circ}$};
\end{tikzpicture}

\Solution

\textit{Geometry Concept}

A rhombus is a parallelogram with four congruent sides. Just like a parallelogram, its two pairs of
opposite angles are congruent and the sum of the measures of any two consecutive angles is 180
degrees.
%
\begin{align*}
6x - 15 &= 4x + 5 && \text{(The given angles are opposite.)}\\
2x &= 20\\
x &= 10\\
4x + 5 &= 45
\end{align*}
The given opposite angles have a measure of 45 degrees each while its other two angles
measure $180 - 45 = 135$ degrees each.

Therefore, the measures of the angles of the rhombus are $45\degree$, $45\degree$, $135\degree$, and $135\degree$.
\end{example}
\subsection*{PROBLEMS INVOLVING COMPLEMENTARY, SUPPLEMENTARY\\ AND CONGRUENT ANGLES}
\begin{example}
\item Given two intersecting lines, find the value of $x$.

\begin{tikzpicture}[scale=2,thick,>=stealth']
\draw [->] (0,0) -- (22.5:1cm);
\draw [->] (0,0) -- (22.5:-1cm);
\draw [->] (0,0) -- (-22.5:1cm);
\draw [->] (0,0) -- (-22.5:-1cm);
\node [right] at ([xshift=5pt]0,0) {$(6x-15)\degree$};
\node [left] at ([xshift=-5pt]0,0) {$(4x+5)\degree$};
\end{tikzpicture}

\Solution

\textit{Geometry Concept}

The two intersecting lines formed angles and the two given angles are “opposite” each other. They are
called \Bold{vertical angles}. Vertical angles are congruent. 

Since the given angles are vertical angle, they are congruent. Refer to Example 8.

Therefore, the value of $x$ is 10.
\end{example}

\begin{example}
\Item Given two intersecting lines, find the value of $x$.

\begin{tikzpicture}[scale=2,thick,>=stealth']
\draw [->] (0,0) -- (22.5:1cm);
\draw [->] (0,0) -- (22.5:-1cm);
\draw [->] (0,0) -- (-22.5:1cm);
\draw [->] (0,0) -- (-22.5:-1cm);
\node [right] at ([xshift=5pt]0,0) {$(6x-15)\degree$};
\node [below] at ([yshift=-3pt]0,0) {$(4x+5)\degree$};
\end{tikzpicture}

\Solution

\textit{Geometry Concept}

The two intersecting lines formed angles and note that when the measures of the two given angles are
added, we get the measure of a straight angle. These two angles are called linear pairs. The degree
measures of a \Bold{linear pair} sum up to 180 degrees.

Since the given angles are linear pairs, they sum up to $180\degree$. Refer to Example 7. %Fix reference here.

Therefore, the value of $x$ is 19.
\end{example}

\begin{example}
\Item The supplement of an angle is 20 degrees more than thrice the measure of the angle. What is the
measure of the angle?

\Solution

\textit{Geometry Concept:} When the measures of two angles sum up to 180 degrees, they are called \Bold{supplementary angles}.
\begin{align*}
x &= \text{the given angle}\\
180 - x &= \text{the measure of the supplement of the angle}\\
180 - x &= 3x + 20\\
160 &= 4x\\
x &= 40
\end{align*}
Therefore, the measure of the angle is 40 degrees.
\end{example}

\begin{example}
\Item The supplement of a certain angle is four times its complement. What is the measure of the angle?

\Solution

\textit{Geometry Concept:} When the measures of two angles sum up to 90 degrees, they are called \Bold{complementary angles}.
\begin{align*}
x &= \text{the given angle}\\
180 - x &= \text{the measure of the supplement of the angle}\\
90 - x &= \text{the measure of the complement of the angle}\\
180 - x &= 4(90 - x)\\
180 - x &= 360 - 4x\\
3x &= 180\\
x &= 60
\end{align*}
Therefore, the measure of the angle is 60 degrees.
\end{example}
\section*{EXERCISES}
\begin{enumerate}
\item If the width of a rectangle is 2 cm more than one-half its length and its perimeter is 40 cm, what are
the dimensions of the rectangle?
\item Each side of a triangle with sides 3 cm, 4 cm and 5 cm is extended the same amount. If the new
perimeter of the triangle is twice its original perimeter, what is the length of each side of the new
triangle?
\item In a triangle, the ratio of the measures of its sides is 2:3:4 and the perimeter is 72 cm. Find the
measures of the sides of the triangle.
\item In a triangle, the ratio of the measures of its angles is 2:3:4. Find the measures of the angles of the
triangle.
\item If the numerical value of the area of a rectangle is twice the numerical value of its perimeter, and
the length of one of its sides is 4.5, what are the lengths of its the other three sides?
\item The measure of an angle is thrice the measure of its supplement. Find the measure of the
supplement of the angle.
\item What is the measure of an angle if the measure of its supplement is 10 degrees more than twice its
complement?
\item The sum of the measures of an acute angle and an obtuse angle is 1400. The sum of twice the
supplement of the obtuse angle and three times the complement of the acute angle is 3400. Find the
measures of the angles.
\item When a beam of light is reflected from a
smooth surface, the angle formed by the
incoming beam with the surface is congruent
to the angle formed by the reflected beam
and the surface.
Given the following
information, the measure of $\angle ABC$ is 90, the
measure of $\angle BCD$ is 75, and the beam of
light makes an angle of $35\degree$ with line segment
$AR$. At what angle does the beam reflect
from line segment $AB$ the second time. You
can indicate computed measures on the
figure.

$m\angle EAF = \makebox[1.5in]{\hrulefill}$

\item The Leaning Tower of Pisa in Italy makes an angle with the ground of
about $84\degree$ on one side the figure and is denoted by $\degree$. Find the measure of
the other angle that the tower makes with the ground.
\end{enumerate}
\section*{ADDITIONAL PROBLEMS}
\begin{enumerate}
\item The design of the patio of Mr. Archangel is shown at the right. If the measure of $\angle1$ is three times as large as the measure of $\angle 2$, calculate the measure of $\angle 1$.
\begin{center}
\begin{tikzpicture}[scale=0.6]
\node (A) [thick,trapezium,trapezium left angle=90,trapezium right angle =45,rotate=90,draw,minimum width=1cm,minimum height=2cm] {};
\draw ([yshift=1pt,xshift=-0.5pt]A.south west) rectangle ++(-10pt,10pt);
\draw ([yshift=1pt]A.north west) rectangle ++(10pt,10pt);
\node at ([xshift=-10pt,yshift=2.2cm]A.south east) {2};
\node at ([xshift=10pt,yshift=-10pt]A.north east) {1};
\end{tikzpicture}
\end{center}
\Solution

We have here a convex quadrilateral. The some of the measures of
the interior angles of a quadrilateral is 360. Two angles are already
right angles, hence, the sum of their measures is 180. The sum of the
measures of $\angle 1$ and $\angle 2$ is 180.

Setting up an equation, we have the following solution:
Let $x$ be the measure of $\angle 2$. The measure of $\angle 1$ is $3x$ and $x+3x=180$
will be our working equation. Simplifying we have $4x=180$ and simplifying further $x = 45$. Therefore the measure of $\angle 1$ is 135.

\item At night time, it is usually warmer during summer at the urban regions that at the rural regions. Part
of the reasons to this is the amount of heat absorbed by the establishments during the day which
they will emit at night time. The narrow city streets and tall building trap and absorbs the sun's heat
during the day. When sunlight shines on flat land, some of the light scatters back into the sky. In
the city, light scattered by the ground or building often hits another building. Instead of escaping
into the atmosphere the heat is absorbed. The more heat is absorbed the greater is the heat
released at night time. The reflection of light is governed by the law of reflection which states that
the angle of incidence is equal to the angle of reflection. The angle of incidence is the angle formed
by the incident ray and the line perpendicular to the surface called the normal line. On the other
hand, the angle of reflection is the angle formed by the reflected ray and the normal line. The figure
below shows two parallel buildings on the opposite sides of the road. $\ray{AB}$ is the incident line and is reflected along the vertical side of the building. It is reflected as $\ray{BC}$, then reflected again as $\ray{CD}$ and finally as $\ray{DE}$. If the measure of $\angle ABG$ is 35, what is the measure of $\angle FDE$?
\begin{center}
\begin{tikzpicture}[>=stealth']
\newcommand{\Value}{\pgfmathparse{atan(2)}\pgfmathresult}
\node (A1) [pattern=bricks,minimum width=0.5cm,minimum height=3cm,draw] {};
\node (A2) at ([xshift=-0.775cm,yshift=-0.5cm]A1.east) [pattern=bricks,minimum width=0.5cm,minimum height=2cm,draw] {};
\node (A3) at ([xshift=-1.275cm,yshift=-1cm]A1.east) [pattern=bricks,minimum width=0.5cm,minimum height=1cm,draw] {};
\begin{scope}[xshift=3cm]
\node (B1) [pattern=bricks,minimum width=0.5cm,minimum height=3cm,draw] {};
\node (B2) at ([xshift=0.25cm,yshift=-0.5cm]B1.east) [pattern=bricks,minimum width=0.5cm,minimum height=2cm,draw] {};
\node (B3) at ([xshift=0.75cm,yshift=-1cm]B1.east) [pattern=bricks,minimum width=0.5cm,minimum height=1cm,draw] {};
\end{scope}
\coordinate (B) at ([yshift=-10pt]B1.north west) {};
\coordinate (C) at ([xshift=20pt]A1.south east) {};
\draw [->] (B) -- (C);
\coordinate (D) at ([yshift=10pt]A1.south east);
\draw [->] (C) -- (D);
\coordinate (E) at ([yshift=30pt]B1.north west);
\draw [shorten >=-20pt,->] (D) -- (E) node [below right] {$E$};
\draw [->] (D) -- +(0,4cm) node [above] {$F$};
\draw (A1.south east) -- (B1.south west);
\draw [->] (B1.south west) -- +(0,4.5cm) node [above] {$G$};
\coordinate (A) at ($(B)!-2cm!96.57:(C)$);
\draw [->] (A) -- (B);
\node [left] at (A) {$A$};
\node [left] at ([xshift=-2pt,yshift=-2pt]B) {$B$};
\node [below] at (C) {$C$};
\node [right] at ([yshift=3pt]D) {$D$};
\useasboundingbox{(0,3cm)};
\end{tikzpicture}
\end{center}
\Solution

We know that the measure of $\angle ABG$ is 35. We are tasked to find the measure of $\angle FDE$. We
also know that the angle of incidence is equal to the angle of reflection. To solve the problem, we
need to redraw the figure and in each reflection process we draw the normal line. We will also use
the theorem complements of congruent angles are congruent. (The normal line is perpendicular to
the surface. The angle of incidence and the angle formed by the incident ray and the surface are
complementary). For the purpose of clearer solution, we will enlarge the figure and remove the
buildings. The broken arrows are the normal line.
\begin{center}
\begin{tikzpicture}[>=stealth']
\newcommand{\Value}{\pgfmathparse{atan(2)}\pgfmathresult}
\node (A1) [minimum width=0.5cm,minimum height=3cm] {};
\begin{scope}[xshift=3cm]
\node (B1) [minimum width=0.5cm,minimum height=3cm] {};
\end{scope}
\coordinate (B) at ([yshift=-10pt]B1.north west) {};
\coordinate (C) at ([xshift=20pt]A1.south east) {};
\draw [->] (B) -- (C);
\coordinate (D) at ([yshift=10pt]A1.south east);
\draw [->] (C) -- (D);
\coordinate (E) at ([yshift=30pt]B1.north west);
\draw [shorten >=-20pt,->] (D) -- (E) node [below right] {$E$};
\draw [->] (D) -- +(0,4cm) node [above] {$F$};
\draw (A1.south east) -- (B1.south west);
\draw [->] (B1.south west) -- +(0,4.5cm) node [above] {$G$};
\coordinate (A) at ($(B)!-2cm!96.57:(C)$);
\draw [->] (A) -- (B);
\node [left] at (A) {$A$};
\node [right] at ([xshift=-2pt,yshift=-2pt]B) {$B$};
\node [below] at (C) {$C$};
\node [right] at ([yshift=3pt]D) {$D$};
\draw [dashed,->] (D) -- +(2cm,0) node [right] {$J$};
\draw [dashed,->] (C) -- +(0,2cm) node [above] {$I$};
\draw [dashed,->] (B) -- +(-2cm,0) node [left] {$H$};
\node at ([xshift=-5pt]B1.south east) {$K$};
\draw ([xshift=-15pt]B1.south east) rectangle +(-5pt,5pt);
\useasboundingbox{(0,3cm)};
\end{tikzpicture}
\end{center}
The measure of $\angle ABG$ is 35. Therefore its complement ($\angle ABH$) has a measure of 55.
Complementary angles are pair of angles whose some of their measures is equal to 90. $\angle ABH$ is the angle of
incidence while $\angle HBC$ is the angle of reflection. By the
law of reflection these angles ($\angle ABH$ and $\angle HBC$) have equal measure. The normal ray ($\ray{BH}$) is parallel to the surface of the road (since the road and the side of the
building are perpendicular and the normal line is
perpendicular to the side of the building. Lines
perpendicular to the same line are parallel). $\Segment{CB}$ is the transversal and $\angle HBC$ and $\angle BCK$ are alternate interior angles. (Alternate interior angles of parallel lines cut by
a transversal are congruent). Therefore the measure of
$\angle BCK$ is 55. The measure of $\angle BCI$ (angle of incidence)
is 35 since $\angle BCK$ and $\angle BCI$ are complementary.

Using the same processes, the measure of $\angle BCI$ is 35, the measure of $\angle ICD$ is 35, $\angle CDJ$ is 55, $\angle JDE$ is also 55 and finally the measure of $\angle FDE$ is 35.

\item Kaitlin and Henry are participating in a treasure hunt. They are on the same straight path,
walking toward each other. When Kaitlin reaches the Big Oak, she will turn $115\degree$ onto another
path that leads to the treasure. At what angle will Henry turn when he reaches the Big Oak to
continue on to the treasure?

\begin{center}
\begin{tikzpicture}[>=stealth']
\node (cloud) [cloud,minimum width=2cm,minimum height=1cm,cloud ignores aspect,draw] {Big Oak};
\draw [->] (cloud) -- +(0.4\linewidth,0) node [right] (B) {};
\draw [->] (cloud) -- +(-0.4\linewidth,0) node [left] (C) {};
\node (D) at ($(cloud)!0.3\linewidth!-115:(C)$) [cloud,minimum width=2cm,minimum height=1cm,cloud ignores aspect,draw] {Treasure};
\draw (cloud) -- (D);
\node [above] at (B) {
\includegraphics[width=0.5in]{henry}
};
\node [above] at (C) {
\includegraphics[width=0.25in]{kaitlin}
};
\end{tikzpicture}
\end{center}

\Solution

To solve the problem, we need to draw a diagram that illustrates the problem. Below is
the diagram that illustrates the problem. We need to assume that Big Oak, Kaitlin and Henry are
on the same line. The angle formed by the rays joining Kaitlin, Big Oak and treasure has a
measure equal to $115\degree$. Therefore we need to get the measure of the supplement of that angle
since it's the angle where Henry will turn if he will reach Big Oak. Hence, Henry should turn at
$65\degree$-angle to continue to the treasure.
\end{enumerate}
\section*{SUGGESTED ACTIVITY}%\pdfmargincomment[author={Joseph S. Tabadero, Jr.},]{Was this copied verbatim from the source? If so, then this has to be paraphrased/re-written.}
This is adopted from Sourcebook on Practical Work for Teacher Trainers: High School Mathematics I and
II Volume 2.
\subsection*{Exploring the Geometry in a Paper Cup}
\subsubsection*{Objectives}
\begin{enumerate}
\item To apply geometric concepts and principles
\item To justify or prove geometric relationships
\end{enumerate}
\subsubsection*{Materials}
\begin{enumerate}
\item 5 pieces of coupon bond
\item 2 square papers
\end{enumerate}
\subsubsection*{Instructional Procedures}
\textit{Introductory Activity}
\begin{enumerate}
\item \textit{Constructing perpendicular lines by paper folding}

Get a piece of paper. Fold the paper to show two (2) creases perpendicular to each other. Call
the fist crease $\Segment{AB}$ and the second crease $\Segment{CD}$. Explain why you think the creases you made are perpendicular.
\item \textit{Constructing perpendicular bisector by paper folding}

Get another piece of paper. Make a crease and call it $\Segment{AB}$. Make another crease $\Segment{CD}$ that bisects $\Segment{AB}$. How do you know that with your procedure the second crease bisects the first? Does the
first crease bisect the second?

\item \textit{Constructing bisector of angles by paper folding}

Get another piece of paper. Make a crease to make an angle with one side on the lower edge of
the paper. Call this angle $\angle BAC$. Did you make an acute, right or obtuse angle? Show how you
will bisect $\angle BAC$ by folding. Call this bisector $\Segment{AD}$.
\end{enumerate}
\textit{Lesson Proper}
\begin{enumerate}
\item \textit{Making a cup}

Let us do an activity on paper folding. Use the square paper and follow the procedure in the
activity sheet that will be distributed.
\item \textit{Post-Activity Discussion}

Let us examine the mathematics in the activity you did. Let us discuss your answers to the
questions in the activity sheet.
	\begin{enumerate}
	\item Which paper folding procedure in our previous activity was used in steps 1, 2 and 3?
	\item Let us fill the table with the names of the polygons found in the figure.
	\item Identify some figures which are symmetrical and give the line of symmetry for each (you can
	omit this since it's not included in the topics to be discussed)
	\item Now we consider the congruent angles and congruent segments that resulted from the
	folding. Use the upper triangular half of the figure only.
	\item List down other possible problems that can be formulated based on the resulting figure in
	the activity
	\end{enumerate}
\end{enumerate}
\textit{Assessment}

Give a piece of square paper. Call the diagonal fold $\Segment{AB}$.
\begin{enumerate}
\item Fold the paper to show an angle of measure $67.5\degree$ with vertex at A.
\item Using the same paper used in number 1, fold the paper to show 2 creases perpendicular to $\Segment{AB}$.
\item Trace the following figures. Name the points as needed and indicate the measure of the angles.
	\begin{enumerate}
	\item Two triangles of different sizes but of the same set of angle measure.
	\item Two triangles that are congruent. Explain why they are congruent.
	\item A trapezoid. Find the measure of its angles and explain why you think it is a trapezoid.
	\end{enumerate}
	If the above questions (assessment of this paper folding activity) are still too high for
	your students, then you can use the following questions instead.
	\begin{enumerate}
	\item What do the creases represent?
	\item Name congruent segments.
	\item Name congruent angles.
	\item Name acute, right and obtuse angles.
	\item Name complementary, supplementary and vertical angles.
	\item Name an angle bisector and the angle it bisects
	\item Name segments which are perpendicular to each other.
	\item Name segments which are parallel.
	\end{enumerate}
\end{enumerate}
\subsubsection*{Making a Paper Cup}
\begin{enumerate}[A.]
\item To make a cup, fold the square paper by following the procedure below.
\begin{center}
\begin{tikzpicture}[thick,>=stealth']
\filldraw [gray, draw=black, opacity=0.5] (0,0) -- (2,2) -- (0,2) -- cycle;
\draw [dashed] (0,0) -- (2,0) -- (2,2);
\path (0,0) -- (2,0) node [pos=0.5,below] {1};
\draw [->] (2,0) arc (0:90:2cm);
\begin{scope}[xshift=2.25cm]
\draw[<->] (0,1) -- ++(7.5pt,-7.5pt) --++(10pt,10pt) --+(7.5pt,-7.5pt);
\node [below=5pt] at ($(0,1)+(12.5pt,0)$) {rotate};
\end{scope}
\begin{scope}[xshift=3.25cm,rotate=-45]
\draw (0,0) -- (2,2) -- (0,2) -- cycle;
\draw [dashed] (0.83,2) -- (0,0);
\path (90:0.5cm) edge [bend left] (45:0.7cm);
\path [->] (45:0.7cm) edge [bend right] (85:0.9cm);
\path (0,0) -- (2,2) node [midway,below] {2};
\end{scope}
\begin{scope}[xshift=6.5cm,rotate=-45]
\draw (0,1.17) -- (0,2) -- (0.83,2) -- cycle;
\draw [dashed] (0,0) -- (0,1.17) -- (45:1.17) -- cycle;
\fill [fill=gray,opacity=.5] (0,1.17) -- (0.83,2) -- (45:1.17) -- cycle;
\draw (0.83,2) -- (2,2) -- (45:1.17) -- cycle;
\path (0,0) -- (2,2) node [midway,below] {3};
\end{scope}
\begin{scope}[shift={(0,-3)},rotate=-45]
\draw (0,1.17) -- (0,2) -- (0.83,2) -- cycle;
\filldraw [fill=gray,opacity=.5] (0,1.17) -- (0.83,2) -- (45:1.17) -- cycle;
\draw [dashed] (0.83,2) -- (2,2) -- (45:1.66) -- cycle;
\path (0,0) -- (2,2) node [midway,below] {4};
\draw (45:1.17) -- (45:1.66);
\fill [fill=gray,opacity=.7] (45:1.66) -- (0,1.17) -- (0.83,2) -- cycle;
\end{scope}
\begin{scope}[shift={(2.5,-3)},rotate=-45]
\draw (0,1.17) -- (0,2) -- (0.83,2) -- cycle;
\filldraw [fill=gray,opacity=.5] (0,1.17) -- (0.83,2) -- (45:1.17) -- cycle;
\path (0,0) -- (2,2) node [midway,below] {5};
\draw (45:1.17) -- (45:1.66);
\filldraw [fill=gray,opacity=.7] (45:1.66) -- (0,1.17) -- (0.83,2) -- cycle;
\filldraw [fill=white] (0,1.17) -- ++(0:.83) --++(90:.83) -- cycle;
\end{scope}
\begin{scope}[shift={(5.25,-2.25)}]
\draw [decoration={coil,aspect=1,segment length=3mm,amplitude=1mm},decorate,->] (0,0) -- (0.8cm,0);
\node at (0.4cm,0) [below=5	pt,align=center] {turn\\ over};
\end{scope}
\begin{scope}[shift={(6,-3)}]
\begin{scope}[rotate=-45]
\filldraw [fill=gray,opacity=.5] (0,1.17) -- (0.83,2) -- (45:1.17) -- cycle;
\path (0,0) -- (2,2) node [midway,below] {6};
\draw (45:1.17) -- (45:1.66);
\filldraw [fill=gray,opacity=.7] (45:1.66) -- (0,1.17) -- (0.83,2) -- cycle;
\filldraw [fill=white] (0,1.17) -- ++(0:.83) --++(90:.83) -- cycle;
\end{scope}[rotate=45]
\filldraw [fill=white] ($([xshift=1cm,yshift=-.7cm]0,1.17)!.5!([xshift=1cm,yshift=-.7cm]0.83,2)$) ellipse (.59 and .2);
\end{scope}
\end{tikzpicture}
\end{center}
\item We study some mathematics related to the figures. carefully unfold the final figure. Label the points as shown.

\begin{center}
\begin{tikzpicture}[scale=2]
\coordinate (A) at (0,0);
\coordinate (B) at (45:2);
\coordinate (C) at (2.83,0);
\coordinate (K) at (-45:2);
\coordinate (E) at (45:1.17);
\coordinate (D) at ($(45:2)+(-45:.83)$);
\coordinate (F) at (1.17,0);
\coordinate (G) at (1.66,0);
\coordinate (H) at (-45:1.17);
\coordinate (J) at ($(-45:2)+(45:.83)$);
\coordinate (L) at ($(E)!.5!(F)$);
\coordinate (M) at ($(F)!.5!(H)$);

\draw (A) -- (B) -- (C) -- (K) -- cycle;
\draw (A) -- (C);
\draw [dashed] (E) -- (D) -- (G) -- (J) -- (H) -- (F) -- cycle;
\draw [dashed] (A) -- (D);
\draw [dashed] (A) -- (J);

\node at (A) [left] {$A$};
\node at (B) [above] {$B$};
\node at (C) [right] {$C$};
\node at (K) [below] {$K$};
\node at (E) [left] {$E$};
\node at (D) [right] {$D$};
\node at (H) [left] {$H$};
\node at (J) [right] {$J$};
\node at ([yshift=-2pt]L) [right] {$L$};
\node at ([yshift=2pt]M) [right] {$M$};
\node at (F) [above left] {$F$};
\node at (G) [above right] {$G$};
\end{tikzpicture}
\end{center}

Answer the following questions. Discuss your answers with your group.

	\begin{enumerate}
	\item Study steps 1 to 3. Which folding in our introductory paper folding procedures correspond to each step? 
	\item What polygons do you recognize? Name some of them and classify them as indicated in the following table.
	\begin{center}
	\begin{tabular}{lcc}
	\hline \hline
	Polygons & Some Convex Polygons & Some Concave Polygons\\
	\hline 
	Triangle & & \\
	Quadrilateral & & \\
	Pentagon & & \\
	Hexagon & & \\
	\hline 
	\end{tabular}
	\end{center}
	\item Study the symmetrical figures. Can you identify symmetrical figures? Name some and identify the line of symmetry for each.
	\item Name all the line segments that are congruent by superimposing. Name also the congruent angles.
	\end{enumerate}
\end{enumerate}