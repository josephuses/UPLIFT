\chapter{SET OF REAL NUMBERS}\label{chap:2}
\section*{INTRODUCTION}
This module is designed to describe, represent and compare the different subsets of real
numbers in hierarchical form. Mainly, this module talks about applications of various procedures and
manipulations on the different subsets of the set of real numbers. The topics included in this module
are: definitions of different kinds of numbers; the location of numbers on the number line;
properties of operations (addition and multiplication) that characterizes each set of numbers;
operations involved in each set of numbers (particularly integers and rational numbers); description
and representation of real-life situations involving integers, rational numbers, square roots of
rational numbers and irrational numbers; and solving problems involving real numbers. An
assessment exam and alternative assessment in terms of group activity are also given at the end of
the module.

\section*{OBJECTIVES}
This module is made towards the goal of helping the learners:
\begin{enumerate}
\item Identify the different sets of numbers using properties of operations (addition and
multiplication).
\item Describe and illustrate the absolute value of a number on a number line as the distance of
the number from 0.
\item Perform fundamental operations on integers: addition, subtraction, multiplication, and
division.
\item Define and illustrate rational numbers and arrange them on a number line.
\item Express rational numbers from fraction form to decimal form (both terminating and
repeating) and vice versa.
\item Perform operations on rational numbers and illustrate their properties.
\item Describe principal roots and tell whether they are rational or irrational.
\item Write in scientific notation.
\end{enumerate}

\section*{DISCUSSION}
\subsection*{Numbers and Numerals}
\Bold{Numbers} are ideas that we associate with quantities of things around us. \Bold{Numerals} are the
symbols we use to represent these ideas. Different people use different symbols to represent
numbers. The \Bold{number line} is a line with points and is often used to illustrate order of numbers
because each point which is associated with, or represents a number has a definite location in
relation to other points that also represent other numbers.
\subsection*{The System of Counting Numbers ($\mathbb{N}$)}
\subsubsection*{$\mathbb{N}$ and the Number Line}
Naturally, the first number we use when we count is the number one (1), followed by two (2),
three (3), and so on. We call these numbers \Bold{counting} or \Bold{natural numbers}. The set of counting
numbers is represented by $N = \{1, 2, 3, 4, 5, 6, 7, \ldots\}$. The number line can be used to show the
correct order of these numbers.
\subsubsection*{Subsets of $\mathbb{N}$}
Some of the subsets of the set of natural numbers are the following:
\begin{enumerate}
\item The set of \Bold{prime numbers}. A \Bold{prime number} is a natural number greater than 1 whose only
positive divisors are 1 and itself. Let us represent the set of prime numbers as set $P$.
\[P=\{2,3,5,7,11,13,17,19,\ldots\}\]
\item The set of \Bold{composite numbers}. A \Bold{composite number} is a natural number that has a positive
other than 1 and itself. In other words, a composite number is a natural number greater than 1
that is \textit{not} a prime number. Let us represent the set of composite numbers as set $C$.
\[C = \{4, 6, 8, 9, 10, 12, 14, 15, 16, \ldots\}\]
\item The set of \Bold{odd numbers}. An \Bold{odd number} is a number that is not divisible by 2. Let us represent
the set of odd numbers as set $O$.
\[O = \{1, 3, 5, 7, 9, 11, 13, 15, 17, \ldots\}\]
\item The set of \Bold{even numbers}. An \Bold{even number} is a number that can be divided by 2. Let us represent
the set of even numbers as set $E$.
\[E = \{2, 4, 6, 8, 10, 12, 14, 16, ...\}\]
\end{enumerate}
\subsubsection*{Number Theory}
The basic number theory concepts or terms included in discussing counting numbers are tDivisibility Tests. The method of determining which counting number is divisible by 2, 3, 4, 5, 6,
7, 8, 9, 10, 11, and other numbers.
he
following.
\begin{enumerate}
\item \Bold{Prime Factors}. These are factors that are prime numbers.
\item \Bold{Prime Factorization}. This is the method of expressing a composite number as a product of
primes.
\item \Bold{GCF and LCM}. The methods of finding the \Bold{greatest common factor (GCF)} and \Bold{least common
multiple (LCM)} of certain sets of numbers.
\item \Bold{Divisibility Tests}. The method of determining which counting number is divisible by 2, 3, 4, 5, 6,
7, 8, 9, 10, 11, and other numbers.
\end{enumerate}
\subsubsection*{Properties of Operations ($+$ and $\times$) in $\mathbb{N}$}
The following properties are always true in $\mathbb{N}$.

\begin{tabularx}{\linewidth}{XC}
Property & Examples\\
1. Closure Property for Addition (CLPA) & $1 + 3 = 4, 4 \in N$\\
2. Closure Property for Multiplication (CLPM) & $4 x 5 = 20, 20 \in N$\\
3. Commutative Property for Addition (CPA) & $3+5=5+3$\\
4. Commutative Property for Multiplication (CPM) & $4\times 8=8\times 4$\\
5. Associative Property for Addition (APA) & $(2 + 5) + 3 = 2 + (5 + 3)$\\
6. Associative Property for Multiplication (APM) & $(4 \times 3) \times 2 = 4 \times (3 \times 2)$\\
7. Distributive Property for Multiplication over Addition (DPMA) & $4 \times (5 + 3) = (4 \times 5) + (4 \times 3)$\\
8. Identity Property of Multiplication (IPM) & $8\times1=1\times8=8$\\
\end{tabularx}

\subsection*{The System of Whole Numbers ($\mathbb{W}$)}
\subsubsection*{$\mathbb{W}$ and the Number Line}
Recall from Module \ref{chap:1} when a set has no element, it is called an \Bold{empty set} or a \Bold{null set}. The cardinal number of
the empty set is represented by a number which is called \Bold{zero (0)}. The union of the unit set $\{0\}$ and the set
$\mathbb{N} = \{1, 2, 3, 4, 5, 6, \ldots\}$ is the set of \Bold{whole numbers} represented by $\mathbb{W} = \{0, 1, 2, 3, 4, \ldots\}$. The set $\mathbb N$
is a subset of the set $\mathbb W$, therefore \Ital{all subsets of $\mathbb N$} (i.e., set of prime numbers, set of composite
numbers, set of even numbers, and set of odd numbers) \Ital{are all subsets of $\mathbb W$}.
\subsubsection*{Properties of Operations ($+$ and $\times$) in $\mathbb W$}
\begin{tabularx}{\linewidth}{XC}
Property & Examples\\
1. Closure Property for Addition (CLPA) & $1 + 3 = 4, 4 \in N$\\
2. Closure Property for Multiplication (CLPM) & $4 x 5 = 20, 20 \in N$\\
3. Commutative Property for Addition (CPA) & $3+5=5+3$\\
4. Commutative Property for Multiplication (CPM) & $4\times 8=8\times 4$\\
5. Associative Property for Addition (APA) & $(2 + 5) + 3 = 2 + (5 + 3)$\\
6. Associative Property for Multiplication (APM) & $(4 \times 3) \times 2 = 4 \times (3 \times 2)$\\
7. Distributive Property for Multiplication over Addition (DPMA) & $4 \times (5 + 3) = (4 \times 5) + (4 \times 3)$\\
8. Identity Property of Multiplication (IPM) & $8\times1=1\times8=8$\\
9. Identity Property of Addition (IPA) & $5 + 0 = 0 + 5= 5$\\
\end{tabularx}

\subsection*{The System of Integers ($\mathbb{Z}$)}
\subsubsection*{$\mathbb{Z}$ and the Number Line}
In the sets $\mathbb N$ and $\mathbb W$, subtraction can only be performed by deducting a smaller number from
a larger number. When a larger number is subtracted from a smaller number, the result is a number
that does not belong to these two sets. Another set of numbers called the \Bold{negative numbers}, whose
values are less than zero, is formed. The union of these negative numbers and $\mathbb W$ is called the set of
integers or signed numbers represented by $\mathbb Z = \{\ldots, -4, -3, -2, -1, 0, 1, 2, 3, 4, \ldots\}$. Since $\mathbb W$ is a subset
of $\mathbb Z$, all subsets of W are also subsets of $\mathbb Z$. The number line shows the position of the points that
represent the integers. The order of the numbers depends on their position relative to the point
which represents zero, also called the \Bold{origin}. The \Bold{negative numbers}, also called the \Bold{additive inverses}
of the natural numbers (\Bold{positive numbers}), are located at the left side of zero, while all positive
numbers are located at the right side of zero. Negative numbers are less than zero, while positive
numbers are greater than zero. Zero is neither positive nor negative.
\subsubsection*{Properties of Operations ($+$ and $\times$) in $\mathbb Z$}
The following properties are true in $\mathbb Z$.

\begin{tabularx}{\linewidth}{XC}
Property & Examples\\
1. Closure Property for Addition (CLPA) & $-2+-4=-6,1 + 3 = 4, -6,4 \in \mathbb Z$\\
2. Closure Property for Multiplication (CLPM) & $-3\times 5=-15,4 x 5 = 20, -15,20 \in \mathbb Z$\\
3. Commutative Property for Addition (CPA) & $-3+5=5+-3$\\
4. Commutative Property for Multiplication (CPM) & $4\times -8=-8\times 4$\\
5. Associative Property for Addition (APA) & $(2 + -5) + 3 = 2 + (-5 + 3)$\\
6. Associative Property for Multiplication (APM) & $(-4 \times 3) \times 2 = -4 \times (3 \times 2)$\\
7. Distributive Property for Multiplication over Addition (DPMA) & $-4 \times (5 + 3) = (-4 \times 5) + (-4 \times 3)$, $(5 + 3) \times -4 = (5 \times -4) + (3 \times -4)$\\
8. Identity Property of Multiplication (IPM) & $-8\times1=1\times8=-8$\\
9. Identity Property of Addition (IPA) & $-5 + 0 = 0 + -5 = -5$\\
10. Additive Inverse Property (AIP) & $5 + (-5) = 0$\\
\end{tabularx}
\subsubsection*{Absolute Value of an Integer}
The absolute value of an integer is the distance of the point representing the integer from
the point representing zero. The absolute value of $+5$, represented by $|+5|$, and the absolute value
of $-5$, represented by $|-5|$, are both equal to 5, since the points representing them are both $5$
units away from the point representing 0.
\subsubsection*{Operations on $\mathbb Z$}
\paragraph*{Addition of Integers}
The sum of two integers depends on the signs of both numbers. If the numbers have the
same sign, the sum is preceded by the common sign. If the numbers have different signs, the
difference of the two numbers is taken and the sign of the answer is the sign of the integer with the
greater absolute value.
\begin{example}
\item 
	\begin{inparaenum}
	\item $(+3)+(+4)=(+7)$\hfil \item $(-3) + (-4) = (- 7)$
	\end{inparaenum}
\item 
	\begin{inparaenum}
	\item $(+3) + (-4) = (- 1)$\hfil \item $(-3) + (+4) = (+1)$
	\end{inparaenum}
\end{example}
\paragraph*{Subtraction of Integers}
Subtraction of integers is the addition of the additive inverse of the subtrahend to the
minuend. In symbols, $a - b = a + (-b)$.
\begin{example}
\item $9-3 = 9 + (-3) = 6$
\item $-9 - 3 = -9 + (-3) = -12$
\item $-9 - (-3) = -9 + -(-3) = -9 + 3 = -6$
\item $9 - (-3) = 9 + -(-3) = 9 + 3 = 12$
\end{example}
\paragraph*{Multiplication of Integers}
The product of two integers with the same sign is positive while the product of two integers
with different signs is negative.

\paragraph*{Division of Integers}
The quotient of two integers with the same sign is positive, while the quotient of two
integers with different signs is negative. Zero can never be a divisor.
\subsection*{The System of Rational Numbers $\mathbb Q$}
\subsubsection*{$\mathbb Q$ and the Number Line}
When an integer is divided by another nonzero integer, and the quotient is an integer having no
remainder, the first integer is said to be divisible by the nonzero integer. But what happens when an
integer is divided by a nonzero integer which is not a divisor or a factor of the first integer? The
quotient is definitely not an integer. Another set of numbers is formed which will satisfy the given
condition, and we call this set as the set of \Bold{fractions}. The fractions can either be \Bold{proper} or \Bold{improper}
fractions. The union of the set of integers and the set of fractions is called the set of \Bold{rational
numbers} which is represented by $\mathbb Q = \{\ldots, -4, -7/2, -3, -5/2, -2, -3/2, -1, -1/2, 0, 1/2, 1, 3/2, 2, 5/2, 3,
7/2, 4, \ldots\}$. In other words, a rational number can be expressed as a ratio or quotient of two integers.
All rational numbers can be represented by fractions. Since the set of integers is a subset of the set
of rational numbers, all subsets of the set of integers are also subsets of set $\mathbb Q$. The number line can
be used to show the location of points represented by the rational numbers. %\pdfmarkupcomment[author=Joseph]{Highlight}{Needs some revision.}
\subsubsection*{Properties of Operations ($+$ and $\times$) in $\mathbb Q$}
The following properties are true in $\mathbb Q$.

\begin{tabularx}{\linewidth}{XC}
Property & Examples\\
1. Closure Property for Addition (CLPA) & $-\frac{1}{2}+-\frac{3}{4}=-\frac{5}{4},1+3=4,-\frac{5}{4},4\in \mathbb{Q}$\\
2. Closure Property for Multiplication (CLPM) & $-\frac{3}{4}\times \frac{2}{3}=-\frac{2}{5},4 x 5 = 20, -\frac{2}{5},20 \in \mathbb Q$\\
3. Commutative Property for Addition (CPA) & $-\frac{3}{2}+\frac{5}{4}=\frac{5}{4}+-\frac{3}{2}$\\
4. Commutative Property for Multiplication (CPM) & $\frac{4}{7}\times 8=8\times\frac{4}{7}$\\
5. Associative Property for Addition (APA) & $(\frac{2}{3}+-\frac{5}{2})+3=\frac{2}{3}+(-\frac{5}{2}+3)$\\
6. Associative Property for Multiplication (APM) & $(-\frac{4}{3}\times 3)\times \frac{2}{5}=-\frac{4}{3}\times (3\times \frac{2}{5})$\\
7. Distributive Property for Multiplication over Addition (DPMA) & $-4\times (\frac{5}{2}+\frac{3}{4})=(-4\times\frac{5}{2})+(-4\times\frac{3}{4})$, $(\frac{5}{2}+\frac{3}{4})\times -4=(\frac{5}{2}\times-4)+(\frac{3}{4}\times-4)$\\
8. Identity Property of Multiplication (IPM) & $-\frac{8}{9}\times 1=1\times-\frac{8}{9}=-\frac{8}{9}$\\
9. Identity Property of Addition (IPA) & $-\frac{5}{7}+0=0+-\frac{5}{7}=-\frac{5}{7}$\\
10. Additive Inverse Property (AIP) & $\frac{5}{9}+(-\frac{5}{9}) = 0$\\
11. Multiplicative Inverse Property (MIP) & $2\frac{1}{2}=1$\\
\end{tabularx}

\subsubsection*{Fractions and Decimals}
A subset of $\mathbb Q$ is the set of fractions. A fraction can be defined as the quotient of a whole
number divided by a natural number. A proper fraction has a numerator less than the denominator,
while an improper fraction has a greater numerator than denominator. Any fraction can be
converted to decimal form, either to a terminating or repeating decimal as the case may be.

To convert a fraction to decimal, the numerator is divided by the denominator. A proper
fraction can be converted to either a terminating or repeating decimal. An improper fraction can be
converted to a decimal form with a whole number part, or to a mixed number form. The mixed
number form is a combination of a whole number and a proper fraction.

To convert a terminating decimal to fraction, the decimal is multiplied by a fraction
equivalent to 1 (whose numerator and denominator is a power of 10) and then simplified by dividing
both numerator and denominator by the greatest common factor. For example, 0.45 will be
multiplied by $\frac{100}{100}$, to get $\frac{45}{100}$, which is further simplified to $\frac{9}{20}$.

To convert a repeating decimal to fraction, the following examples can be followed.

\paragraph*{Example 1.} Convert $0.333\ldots$ to fraction.
\paragraph*{Solution.} 
\begin{tabular}{r@{}l}
Let $n\,$ & $=0.333\ldots$\\
$10n\,$ & $=3.333\ldots$\\ \hline
$9n\,$ & $=3$\\
$\therefore n\,$ & $=\frac{1}{3}$\\
\end{tabular}

\paragraph*{Example 2.} Convert $0.4545\ldots$ to a fraction.
\paragraph*{Solution.}
\begin{tabular}{r@{}l}
Let $n\,$ & $= 0.4545\ldots$\\
 $100n\,$ & $= 45. 4545\ldots$\\
  $- n\,$ & $= 0.4545\ldots$\\ \hline
  $99n\,$ & $= 45$\\
  $\therefore n\,$ & $=\frac{5}{11}$\\
\end{tabular}

\paragraph*{Example 3.} Convert $4.3636\ldots$ to a fraction.
\paragraph*{Solution.}
\begin{tabular}{r@{}l}
 Let $n\,$ & $= 4.3636\ldots$\\
  $100n\,$ & $= 436.3636\ldots$\\
   $- n\,$ & $= 4.3636\ldots$\\ \hline
    $99n\,$ & $= 432$\\
   $\therefore n\,$ & $=\frac{48}{11}$\\
\end{tabular}

\subsubsection*{Operations on Natural Numbers}
\paragraph*{Addition and Subtraction}
Rational numbers in fraction form can only be added or subtracted if they have the same
denominators. These fractions are called similar fractions. Always convert dissimilar fractions to
fractions with the same denominators first before adding or subtracting.

\paragraph*{Multiplication}
When multiplying fractions, cancellation can be used to reduce the product to its simplest
form. Cancellation is done by dividing the numerator and denominator by the greatest common
factor.

\paragraph*{Division}
Division of a fraction by another fraction is done by multiplying the dividend by the
reciprocal or multiplicative inverse of the divisor. Then the product is reduced to its simplest
form.

\subsection*{The System of Irrational Numbers $\mathbb{Q'}$}
\subsubsection*{The Set of $\mathbb{Q'}$ and the Number Line}
The numbers that cannot be represented as a ratio of two integers in simplest form are
called irrational numbers. The union of the set of rational numbers and the set of irrational
numbers comprise the set of real numbers which is represented by $\Re$. The set of irrational
numbers is the complement of the set of rational numbers in the universal set of real numbers,
and is represented by
\[\mathbb{Q'}=\{\ldots \sqrt{2},\sqrt{3},\sqrt{5},\sqrt{7},\sqrt{11},\pi,\ldots\}.\]

\subsubsection*{Square Root of a Number as a Rational Number or Irrational Number}
Square roots of numbers sometimes result in rational or irrational numbers.

The square roots of perfect squares, or of fractions having perfect squares in both the
numerator and denominator, are rational numbers.
\begin{example}
\item $\sqrt{49}=\sqrt{7^2}=7$
\item $\sqrt{121}=\sqrt{11^2}=11$
\end{example}
Otherwise, the square roots are irrational numbers.
\subsection*{Scientific Notation}
When writing very large or very small numbers, like the mass of the earth, (6 000
000 000 000 000 000 000 000 kilograms) or the mass of an electron (0.000 000 000 000 000
000 000 000 000 000 911 kilogram), the most convenient form to use is the scientific
notation. In this notation, numbers are represented by the product of a multiplying factor
and a power of ten. A power of ten is the number 10, raised to an integral exponent. The
decimal point is moved until only one nonzero digit is on the left side of it. Then the number
of places the decimal point is moved is used as the exponent of 10. The sign of the exponent
depends on the movement of the decimal point: to the right ($-$) or to the left ($+$).
\begin{example}
\item $6\, 000\, 000\, 000\, 000\, 000\, 000\, 000\, 000\, = 6.0 \times 10^{24}$
\item $0.000\, 000\, 000\, 000\, 000\, 000\, 000\, 000\, 000\, 000\, 911 = 9.11 x 10^{-31}$
\end{example}

\section*{EXERCISES}
\subsection*{Tests for Divisibility.}
Draw {\large\Smiley{}} on the third column if the number in the first column is divisible by the number on the
second column. Otherwise, draw {\large\Frowny{}}.

\noindent
\begin{tabular}{lcl}
1. 67348 & 4 & \hphantom{{\Large\Frowny} answer} \\ \cline{3-3}
2. 73488 & 8 & \\ \cline{3-3}
3. 83671 & 3 & \\ \cline{3-3}
4. 5334 & 6 & \\ \cline{3-3}
5. 5991 & 6 & \\ \cline{3-3}
6. 4425575 & 9 & \\ \cline{3-3}
7. 47830 & 5 & \\ \cline{3-3}
8. 53867 & 11 & \\ \cline{3-3}
9. 1183 & 7 & \\ \cline{3-3}
10. 13855 & 17 & \\ \cline{3-3}
\end{tabular}

\subsection*{Properties of Real Numbers}
\begin{enumerate}
\item How many prime numbers are there from 1 to 50? From 51 to 100? From 101 to 150? From 151
to 200? (Use Sieve of Eratosthenes)
\item How many factors does 72 have? 120? 144? 84? 360?
\item Express each of the following numbers in scientific notation:
	\begin{enumerate}
	\item 45 000 000 000 000 000 000 000 000 000 000
	\item 0.000 000 000 000 000 173
	\item 930 000 000 000 000 000 000
	\item 0.000 000 000 000 000 000 000 000 62
	\item 508 000 000 000 000 000
	\end{enumerate}
\end{enumerate}

\section*{SUGGESTED ACTIVITIES}
\subsection*{Activity 1: Illustrating the Set of Numbers using Venn Diagrams}
\begin{enumerate}
\item Illustrate the relationship of each of the following sets of numbers using the Venn
Diagrams:
	\begin{enumerate}
	\item Prime Numbers and Composite Numbers
	\item Prime Numbers and Even Numbers
	\item Composite Numbers and Odd Numbers
	\item Composite Numbers and Even Numbers
	\item Natural Numbers and Whole Numbers
	\end{enumerate}
\item Show the Hierarchy of the System of Real Numbers using the following:
	\begin{enumerate}
	\item Venn Diagram
	\item Flow Chart
	\item Concept Map
	\end{enumerate}
\end{enumerate}

\subsection*{Activity 2: Categorizing Numbers into the Different Sets}
Check the appropriate boxes that categorize the given numbers.
{\renewcommand{\arraystretch}{1.5}
\begin{center}
\begin{tabular}{|l|c|c|c|c|c|c|}
\hline 
Number & $\mathbb N$ & $\mathbb W$ & $\mathbb Z$ & $\mathbb Q$ & $\mathbb{Q’}$ & $\Re$ \\ \hline \hline
$45.9$ & \hphantom{check} & \hphantom{check} & \hphantom{check} & \hphantom{check} & \hphantom{check} & \hphantom{check} \\ \hline
$-1\,444$  & & & & & & \\ \hline
$3\sqrt{6}$  & & & & & & \\ \hline
$\frac{5}{9}$  & & & & & & \\ \hline
$125\%$  & & & & & & \\ \hline
$27.5$  & & & & & & \\ \hline
$3\, 678\, 125$  & & & & & & \\ \hline
$5\sqrt{2693}$  & & & & & & \\ \hline
$8.246$  & & & & & & \\ \hline
$7\sqrt{9}$  & & & & & & \\ \hline
\end{tabular}
\end{center}
}

\subsection*{Activity 3: The Boat is Sinking (Divisibility Version)}
\subsubsection*{Objective:}
To practice divisibility rules in a fun way.
\subsubsection*{Preparation:}
Each participant is given a sheet of paper (can be a scratch paper) and they should write
their favourite one digit on the sheet of paper (it should be large enough to occupy the
whole sheet).
\subsubsection*{Activity:}
\begin{enumerate}
\item The resource person calls on the participants (just like in the party game, “The Boat is
Sinking”) to group themselves (using their own one-digit numbers as digit) to form numbers
like:
	\begin{enumerate}
	\item two-digit numbers divisible by 2
	\item three-digit numbers divisible by 3
	\item four-digit numbers divisible by 4
	\item five-digit numbers divisible by 5
	\end{enumerate}
\item Participants that cannot form groups will be disqualified from the activity. After some time,
remaining participants are declared winners.
\end{enumerate}
\subsection*{Activity 4: Words of Wisdom}
\subsubsection*{Objectives:}
To be able to conduct a drill on arithmetic (operations on integers).
\subsubsection*{Preparation:}
Each participant is given a copy of the worksheet.
\subsubsection*{Activity:}
\begin{enumerate}
\item After each participant has a copy of the worksheet, the resource person announces the
operation for each column: Col A ($\times$), Col B ($-$), Col C ($+$) and Col D ($\div$).
\item The worksheet is quite self-explanatory. If they have found “the phrase” and they are
satisfied, they may ask the resource person if they are correct.
\item After the allotted time, the worksheet’s answers are revealed.
\end{enumerate}
%\newpage
\vfill 
\def\Cline{\cline{1-1} \cline{3-6} \cline{8-11} \cline{13-16} \cline{18-21}}
\begin{center}
{\LARGE\bfseries WORDS OF WISDOM (\textit{an arithmetic drill})}\\[24pt]
{\Large\bfseries PART 1: BASIC ARITHMETIC}\\[5pt]
Using the indicated operation, compute as fast as you can but do not be careless.\\
\begin{tabular}{|c|@{}c@{}|c|c|c|c|@{}c@{}|c|c|c|c|@{}c@{}|c|c|c|c|@{}c@{}|c|c|c|c|}
\Cline
Row & \hphantom{=} & \multicolumn{2}{c|}{} &  & Col A & \hphantom{=} & \multicolumn{2}{c|}{} & & Col B & \hphantom{=} & \multicolumn{2}{c|}{} &  & Col C & \hphantom{=} & \multicolumn{2}{c|}{} & & Col D\\ \Cline
1 & & 1 & 6 & = & & & 2 & 1 & = & & & 3 & 6 & = & & & 21 & 7 & = & \\ \Cline
2 & & 2 & 7 & = & & & 3 & 6 & = & & & 2 & 1 & = & & & 36 & 6 & = & \\ \Cline
3 & & 3 & 8 & = & & & 2 & 7 & = & & & 4 & 6 & = & & & 27 & 3 & = & \\ \Cline
4 & & 2 & 2 & = & & & 4 & 8 & = & & & 7 & 7 & = & & & 48 & 8 & = & \\ \Cline
5 & & 4 & 4 & = & & & 7 & 2 & = & & & 8 & 8 & = & & & 72 & 9 & = & \\ \Cline
6 & & 7 & 3 & = & & & 8 & 4 & = & & & 6 & 2 & = & & & 84 & 4 & = & \\ \Cline
7 & & 8 & 5 & = & & & 6 & 3 & = & & & 5 & 4 & = & & & 63 & 7 & = & \\ \Cline
8 & & 6 & 9 & = & & & 5 & 5 & = & & & 3 & 3 & = & & & 55 & 5 & = & \\ \Cline
9 & & 4 & 6 & = & & & 3 & 9 & = & & & 1 & 5 & = & & & 39 & 3 & = & \\ \Cline
10 & & 3 & 1 & = & & & 1 & 6 & = & & & 2 & 9 & = & & & 16 & 2 & = & \\ \Cline
\end{tabular}

\vspace{12pt}
{\Large\bfseries PART 2: FROM NUMBERS TO LETTERS TO\\[5pt] WORDS}\\[5pt]
From Part 1, compute from the corresponding column and row.\\
Convert the value to the corresponding letter in the standard English alphabet. (e.g. 1 $\rightarrow$ A, 2 $\rightarrow$ B)
Then, rearrange the letters to form a word.\\[24pt]

\def\CLine{\cline{1-4} \cline{6-9} \cline{11-14} \cline{16-19}}
\begin{tabular}{|c|@{}c@{}|c|c|@{}c@{}|c|@{}c@{}|c|c|@{}c@{}|c|@{}c@{}|c|c|@{}c@{}|c|@{}c@{}|c|c|}
\CLine
\multicolumn{4}{|c|}{WORD 1} & \hphantom{=} & \multicolumn{4}{c|}{WORD 2} & \hphantom{=} & \multicolumn{4}{c|}{WORD 4} & \hphantom{=} & \multicolumn{4}{c|}{WORD 6}\\ \CLine
A4+D6 & = & \hphantom{NO} & \hphantom{NO} & & C7/C1 & = & \hphantom{NO} & \hphantom{NO} & & A1$\star$D1 & = & \hphantom{NO} & \hphantom{NO} & & D7+D8 & = & \hphantom{NO} & \hphantom{NO}\\ \CLine
A2+C8 & = & \hphantom{NO} & \hphantom{NO} & & D6/C2 & = & \hphantom{NO} & \hphantom{NO} & & C3-B3 & = & \hphantom{NO} & \hphantom{NO} & & B7/C2 & = & \hphantom{NO} & \hphantom{NO}\\ \CLine
B1$\star$C1 & = & \hphantom{NO} & \hphantom{NO} & & B2$\star$B9 & = & \hphantom{NO} & \hphantom{NO} & & A9-C6 & = & \hphantom{NO} & \hphantom{NO} & & A10-B4 & = & \hphantom{NO} & \hphantom{NO}\\ \CLine
A1+D4 & = & \hphantom{NO} & \hphantom{NO} & & A6-A5 & = & \hphantom{NO} & \hphantom{NO} & & A7$\star$D5 & = & \hphantom{NO} & \hphantom{NO} & & C10-C8 & = & \hphantom{NO} & \hphantom{NO}\\ \CLine
B4+D9 & = & \hphantom{NO} & \hphantom{NO} & & B6/C5 & = & \hphantom{NO} & \hphantom{NO} & & C4+D3 & = & \hphantom{NO} & \hphantom{NO} & & A10$\star$C9 & = & \hphantom{NO} & \hphantom{NO}\\ \CLine
D5/B6 & = & \hphantom{NO} & \hphantom{NO} & & WORD & \multicolumn{3}{c|}{} & & WORD & \multicolumn{3}{c|}{} & & WORD & \multicolumn{3}{c|}{}\\ \CLine
B7/D4 & = & \hphantom{NO} & \hphantom{NO} & \multicolumn{15}{c}{}\\ \cline{1-4} \cline{6-9} \cline{11-14}
A9+B10 & = & & & & \multicolumn{4}{c|}{WORD 3} & & \multicolumn{4}{c|}{WORD 5} & \multicolumn{5}{c}{}\\ \cline{1-4} \cline{6-9} \cline{11-14}
D7-B3 & = & & & & D10-B8 &=& & & & C6+C10 &=& & & \multicolumn{5}{c}{}\\ \cline{1-4} \cline{6-9} \cline{11-14}
B5$\star$C2 & = & & & & A8/C9 &=& & & & A3-D8 &=& & & \multicolumn{5}{c}{}\\ \cline{1-4} \cline{6-9} \cline{11-14}
C3+D2 & = & & & & A4$\star$B5 &=& & & & A6-D2 &=& & & \multicolumn{5}{c}{}\\ \cline{1-4} \cline{6-9} \cline{11-14}
A3+B10 & = & & & & A2+C7 &=& & & & B1$\star$D1 &=& & & \multicolumn{5}{c}{}\\ \cline{1-4} \cline{6-9} \cline{11-14}
C4-D3 & = & & & & WORD & \multicolumn{3}{c|}{} & & C6+C10 &=& & & \multicolumn{5}{c}{}\\ \cline{1-4} \cline{6-9} \cline{11-14}
C2$\star$C9 & = & & & \multicolumn{5}{c}{} & & WORD & \multicolumn{3}{c|}{} & \multicolumn{5}{c}{}\\ \cline{1-9} \cline{11-14}
WORD & \multicolumn{8}{c|}{} & \multicolumn{10}{c}{}\\ \cline{1-9}
\end{tabular}

\vspace{12pt}
{\Large\bfseries PART 3: FROM WORDS TO A PHRASE}\\[5pt]

Rearrange the words in part 2 to form a phrase.\\[12pt]

\hrulefill
\end{center}
