\chapter[SOLVING LINEAR EQUATIONS AND LINEAR INEQUALITIES]{SOLVING\\ LINEAR EQUATIONS AND\\ LINEAR INEQUALITIES}
\section*{INTRODUCTION}
This section covers equalities and inequalities in one variable (also known as linear equations
and linear inequalities). Equations and inequalities are relations between two quantities. Many real-life situations can be modeled by equations and inequalities. Learning how to solve these
mathematical models enables one to use mathematical solutions to answer real-world problems.
\section*{OBJECTIVES}
\begin{enumerate}
\item Differentiate equations from inequalities.
\item Find the solution of an equation and inequality involving one variable
\begin{enumerate}
	\item from a given replacement set;
	\item intuitively by guess and check;
	\item by algebraic procedures (applying the properties of equalities and inequalities); and/or
	\item by graphing.
\end{enumerate}
\item Check a solution of an equation and inequality.
\item Solve problems involving equations (separate module) and inequalities.
\end{enumerate}
\section*{DISCUSSION}
The mathematical sentence “$4 + 3 = 7$” is true while “$3 > 4$” is false. On the other hand, the
sentences “$x - 3 = 5$” and “$x + 4 > 1$” cannot be determined as true or false until we have a
replacement for the variable $x$. The sentences “$x - 3 = 5$” and “$x + 4 > 1$” are examples of \Bold{open
sentences}. An open sentence of the first degree in one variable can either be an \Bold{equation} or an
\Bold{inequality}. First degree equations and inequalities in one variable are known as \Bold{linear equations} or
\Bold{linear inequalities}.

An \Bold{equation} is a statement that expresses the equality of two mathematical expressions. A lunear
equation is of the form $\mathbf{ax + b = c}$ where $a$, $b$ and $c$ are any real numbers. To solve an equation is to
find a value of the variable that makes the equation true. This value which can replace the variable in
the equation is called a \Bold{root} or a \Bold{solution} of the equation. For example, 2 is a solution of the
equation $2x - 3 = 1$ because when 2 is substituted to $x$, $2(2) - 3 = 1$ which may be simplified to the
true statement $1 = 1$.

The \Bold{replacement set} or \Bold{domain} is the set of possible values that can be used in place of the variable
in an open sentence. A collection of all solutions or all elements of the replacement set of an
equation is called the \Bold{solution set} of the equation. Equations having the same solution are said to be
\Bold{equivalent equations}.

Solving equations and inequalities can be done in four ways: a) from a given \Bold{replacement set};
b) \Bold{guess and check}--one guesses and substitutes values into the equation then check if a true
statement will result; c) \Bold{algebraic procedures}; and d) \Bold{graphing}. Since there is only one or unique
solution of an equation, the graphical method (graphical method) is more emphasized in solving
inequalities.

\begin{property}[frametitle={Properties of Equality}]
\begin{itemize}
\item \Bold{Reflexivity}: For any real number $a$, $a
= a$; that is, anything is equal to itself.
\item \Bold{Symmetry}: For any real numbers $a$ and $b$, if $a=b$, then $b=a$.
\item \Bold{Transitivity}: For any real numbers $a$, $b$ and $c$, if $a=b$ and $b=c$, then $a=c$.
\item \Bold{Adition Property of Equality} \Bold{(APE)}: Any real number added to both sides of an
equation does not change the nature of the equation. That is, if $a$, $b$ and $c$ are real numbers, and $a = b$, then $a + c = b + c$.
\item \Bold{Multiplication Property of Equality} \Bold{MPE}: ny real number multiplied to both sides
of an equation does not change the nature of the equation. That is, if $a$, $b$ and $c$ are real numbers, and $a = b$, then $ac = b c$.
\end{itemize}
\end{property}
\begin{example}
\Item Find the solution set of $2 - 5z = 7$. The replacement set is $\{-1, 0, 1\}$. \hfill \textbf{(From a given replacement set)}

\Solution

\begin{tabularu}{>{$}l<{$}>{$}l<{$}>{$}l<{$}}
\text{Replace $z$ with $-1$}. & \text{Replace $z$ with $0$}. & \text{Replace $z$ with $1$}.\\
2-5z = 7 & 2-5z = 7 & 2-5z = 7\\
2-5(-1) = 7? & 2-5(0) = 7? & 2-5(1) = 7?\\
2+5=7? & 2-0=7? & 2-5=7?\\
7=7\quad \mathbf{True} & 2=7\quad \mathbf{False} & -3 = 7\quad \mathbf{False}\\
\end{tabularu}

The value of $-1$ for $z$ makes $2 - 5z = 7$ true. The solution set is $\{-1\}$.

When no number from the replacement set makes the open sentence true, then the solution set is null.

\Item Check whether 0, 1, 4 and 6 are solutions of $3x-15 = 3.$ \hfill \textbf{(By guess and check.)}

\Solution

\begin{tabularu}{>{$}l<{$}>{$}l<{$}l}
x=0: & 3(0)-15=3? & No, 0 is not a solution of the given equation because $-15\neq 3$.\\
x=1: & 3 (1)-15 = 3? & No, 1 is not a solution because $-12\neq3$\\
x=4: & 3 (4)-15 = 3? & No, 4 is not a solution because $-3\neq 3$\\
x=6: & 3 (6)-15 = 3? & Yes, 6 is a solution of the equation because $3 = 3$.
\end{tabularu}
\Item Solve the following equations: \hfill \textbf{(By algebraic procedure)}
	\begin{enumerate}[1.]
	\item $2x+1=2x+3$
	
	Since the terms $2x$ of the equation cancels out after adding $-2x$ on both sides, the equation simplifies to $1 = 3$, which is a false statement. This equation is an example of a \Bold{contradiction}, an equation that cannot be true for any value of the variable. \underline{Its solution set is the null set}.

	\item $2x + 1 = x + 3$
	
	Upon solving, the value $x = 2$ makes the equation true. This linear equation is an example of a \Bold{conditional equation}, an equation which is true only for a particular value of the variable. For linear conditional equations, exactly one real number will satisfy it. \underline{Therefore, its solution set is a unit set or a singleton (a one-element set)}.
	
	\item $2x + 1 = 2x + 1$
	
	Since the terms $2x$ of the equation cancels out after adding $-2x$ on both sides, the equation simplifies to $1 = 1$, which is a true statement. This equation is an example of an identity, an equation that is true for any value of the variable. Its solution set is the set of real numbers.
	\end{enumerate}
\end{example}
\subsection*{Exercises}
\begin{enumerate}[I.]
\item Find the solution set of each equation if the replacement set is $\{-2, 0, 2\}$.
	\begin{enumerate}[1.]
	\item $x+3=1$
	\item $5-2y=1$
	\item $x=-x$
	\item $\frac{3y}{2}+\frac{1}{2}=2$
	\item $4t-4(t-3) = 12$
	\item $-2(z-5) = 8-2z$
	\end{enumerate}
	\rotatedbox{%
	\begin{inparaenum}[1.]
	\item $\{-2\}$
	\item $\{2\}$
	\item $\{0\}$
	\item $\{\}$
	\item $\{-2,0,2\}$
	\item $\{\}$
	\end{inparaenum}		
	}

\item Find the solution set of the following equations and classify if it is a contradiction, conditional or identity.
	\begin{enumerate}[1.]
	\item $x+3=1$
	\item $5-2y=1$
	\item $x=-x$
	\item $\frac{3y}{2}+\frac{1}{2}=2$
	\item $4t-4(t-3) = 12$
	\item $-2(z-5) = 8-2z$
	\end{enumerate}
	\rotatedbox{%
	\begin{inparaenum}[1.]
	\item $\{-2\}$ conditional
	\item $\{2\}$ conditional
	\item $\{0\}$ conditional
	\item $\{1\}$ conditional
	\item $\mathbb{R}$ identity
	\item $\{\}$ contradiction
	\end{inparaenum}
	}

\item Give the solution sets of the following equations.
	\begin{enumerate}[1.]
	\item $3x +4 = 2(x - 6) + (x + 16)$
	\item $2.3 x-2(x + 1) = 0.7$
	\item $5x + 7 = 7x + 5 – (2x + 2)$\\
	\item $8(3p-1)-5(3p + 2) = 3(2p-1)$
	\item $\frac{x+1}{3}-\frac{x+2}{3}=\frac{x}{2}$
	\item $2(x + 1) + 7(x-1) = 3(x + 1) + 6(x-1)$
	\end{enumerate}
	\rotatedbox{
	\begin{inparaenum}[1.]
	\item $\mathbb{R}$
	\item $\{9\}$
	\item $\{\}$
	\item $\{5\}$
	\item $\{-\frac{2}{3}\}$
	\item $\{\}$
	\end{inparaenum}		
	}
\end{enumerate}

\subsection*{GRAPHING EQUATIONS AND INEQUALITIES}
A \Bold{number line} represents the order of real numbers. Also, real numbers can be represented as points in the number line. On the number line shown in Figure \eqref{chap6fig:1}, point $A$ is the graph of $-3$.
\begin{figure}[!h]
\centering
\caption{A point on the number line.}
\tikz [scale=0.8] {
\draw [<->] (-8.5,0) -- (6.5,0);
\foreach \x in {-8,-7,-6,...,4,5,6}
	\draw (\x,2pt) -- (\x,-2pt) node [below] {$\x$};
\node [above] at (-3,0) {$A$};
\fill [blue,opacity=.5] (-3,0) circle (3pt);
}
\label{chap6fig:1}
\end{figure}

If a number is found on the left of another number on the number line, it is always less than the number on its right. For example, $-5$ is less than $-4$ because $-5$ is on the left of $-4$.

\subsection*{Exercises}
Graph these numbers on the number line.

\begin{inparaenum}
\item 2.5 \hfil \item $-1\frac{1}{2}$ \hfil \item $\frac{5}{4}$ \hfil \item $-3\frac{2}{3}$
\end{inparaenum}

Draw the graph of the following.

\begin{inparaenum}
\item $x<4$ \hfil \item $x\ge\frac{1}{2}$ \hfil \item $x\neq 1$ \hfil \item $x\le \frac{1}{3}$
\end{inparaenum}

\subsection*{INEQUALITIES IN ONE VARIABLE}
Another type of an open sentence is an inequality. A linear inequality is of the form $ax + b < c$ or
$ax + b > c$ or $ax + b\le c$ or $ax + b \ge c$ where $a$, $b$ and $c$ are real numbers. Take note of the inequality symbols and their meanings.

\begin{table}[!h]
\centering
\caption{Inequality symbols and their meaning}
\begin{tabular}{>{$}c<{$}l}
\hline \hline
\text{Inequality Symbol} & Meaning\\
< & less than\\
> & greater than\\
\le & less than or equal to\\
\ge & greater than or equal to\\
\neq & not equal to\\
\hline 
\end{tabular}
\label{chap6tab:1}\eqref{chap6tab:1}
\end{table}

A solution of an inequality is a number that satisfies the inequality statement when it is substituted. For example, 1 is a solution of $x + 1 < 6$ because $2 < 6$ is a true statement. But 1 is not the only solution to $x + 1 < 6$. We can have 2, 3, 4, 5 or 2.2 because all these make the
inequality statement true. We can write the \Ital{solution} as $x < 5$.

\begin{property}[frametitle={Trichotomy Property}]
If $a$ and $b$ are real numbers, then one and only one of the following is true:
\begin{itemize}
\item \begin{tabular}[b]{cc}
$a>b$ & \tikz [baseline=(current bounding box.east)] {
\draw [xscale=-0.5,->] (0,0) -- (1,0);
\draw  (0,0) -- (1,0);
\draw [xscale=1.5,->] (0,0) -- (1,0);
\foreach \x/\xtext in {0/a,1/b}
	\draw (\x,2pt) -- (\x,-2pt) node [below] {$\xtext$};
} \\
\end{tabular}
\item \begin{tabular}[b]{cc}
$a=b$ & \tikz [baseline=(current bounding box.east)] {
\draw [xscale=-0.5,->] (0,0) -- (1,0);
\draw  (0,0) -- (1,0);
\draw [xscale=1.5,->] (0,0) -- (1,0);
\draw (0.5,-2pt) -- (0.5,2pt);
\node [below] (0,0) {$b$};
\node [above] (0,0) {$a$};
} \\
\end{tabular}
\item \begin{tabular}[b]{cc}
$a>b$ & \tikz [baseline=(current bounding box.east)] {
\draw [xscale=-0.5,->] (0,0) -- (1,0);
\draw  (0,0) -- (1,0);
\draw [xscale=1.5,->] (0,0) -- (1,0);
\foreach \x/\xtext in {0/b,1/a}
	\draw (\x,2pt) -- (\x,-2pt) node [below] {$\xtext$};
} \\
\end{tabular}
\end{itemize}
\end{property}
Solving inequalities follows the same procedure as solving equations. The same properties hold except for the multiplication property. A negative real number multiplied to both sides of an inequality \Ital{reverses} the \Ital{direction} of the inequality. Why is this so?

Let's take a look. Try multiplying or dividing the statement below by a negative number, say $-2$.
\begin{align*}
2&<3 && \text{true statement}\\
-2(2) &\overset{?}{<}-2(3) && \text{multiply by $-2$}\\
-4&\overset{?}{-6} && \text{Is this statement true?}\\
 & && \text{\textbf{No, $\mathbf{-4}$ is greater than $\mathbf{-6}$}}\\
-4 &\overset{\checkmark}{>}-6 && \text{To make the statement true, reverse the sign.}
\end{align*}
Let’s summarize the properties of inequality.
\begin{property}[frametitle={Properties of Inequality}]
The following \Bold{properties of inequality} hold for any real numbers $a$, $b$ and $c$.
\begin{itemize}
\item \Bold{Transitivity}: If $a<b$ and $b<c$, then $a<c$.
\item \Bold{Addition Property of Inequality} (\Bold{API}): If $a<b$, then $a+c<b+c$.
\item \Bold{Multiplication Property of Inequality} (\Bold{MPI}): If $a<b$ and $c>0$ ($c<0$), then $ac<bc$ ($ac>bc$).
\end{itemize}
\end{property}
The solution of inequalities is more concretized when it is simultaneously presented with a graph.

The following are graphing symbols that will be used in solving inequalities.
\begin{property}[frametitle={Graphing Symbols}]
\begin{tabular}[t]{lp{0.7\linewidth}}
\tikz [baseline=(current bounding box.east),very thick,>=stealth'] {
\node (a) [circle, minimum width=6pt,draw,inner sep=0pt] {};
\draw [->] (a) -- +(1,0);
} & 
  greater than (the open circle indicates that this is \Ital{not} equal to the number graphed)\\
\tikz [baseline=(current bounding box.east),very thick,>=stealth'] {
\node (a) [circle, minimum width=6pt,draw,fill,inner sep=0pt] {};
\draw [->] (a) -- +(1,0);
} & greater than or equal to (the closed circle indicates that this is equal to the number graphed)\\
\tikz [baseline=(current bounding box.east),very thick,>=stealth'] {
\node (a) [circle, minimum width=6pt,draw,inner sep=0pt] {};
\draw [->] (a) -- +(-1,0);
} & less than\\
\tikz [baseline=(current bounding box.east),very thick,>=stealth'] {
\node (a) [circle, minimum width=6pt,draw,fill,inner sep=0pt] {};
\draw [->] (a) -- +(-1,0);
} & less than or equal to\\
\end{tabular} 
\end{property}
\begin{example}[I.]
\Item Solve the following inequalities then graph the solution set on a number line:
	\begin{enumerate}[1.]
	\item $3x + 2<2$
	
	\Solution
	{\setlength{\abovedisplayskip}{0pt}
	\begin{align*}
	3x+2+(-2)&<2+(-2) && \text{API (add $-2$ to both sides)}\\
	3x+0&<0 && \text{Additive inverse property}\\
	3x&<0 && \text{Additive identity property}\\
	\Bigl(\frac{1}{3}\Bigr)3x&<\Bigl(\frac{1}{3}\Bigr)(0) && \text{MPI}\\
	x&<0 && \text{Zero property}
	\end{align*}}
	We can give the solution set as $x<3$ or as $\{x\vert x<3\}$ and we can graph this as in \eqref{chap6fig:2}.
	\begin{figure}[!h]
	\centering
	\caption{Solution of $x<3$}
	\begin{tikzpicture}[>=stealth']
	\foreach \x in {-5,-4,...,4,5}
		\draw (\x,2pt) -- (\x,-2pt) node [below] {$\x$};
	\draw [<->, thick] (-5.5,0) -- (5.5,0);
	\node (a) [circle, ultra thick, inner sep=0pt, blue, opacity=0.5, text width=5pt, draw] {};
	\draw [->, ultra thick,blue,opacity=0.5] (a) -- +(-5,0);
	\end{tikzpicture}
	\label{chap6fig:2}
	\end{figure}
	The open circle on 0 indicates that 0 is not part of the solution set.

\CHECK: Take a test value on the interval, say $x = -1$.
\begin{align*}
3x+2&<2\\
3(-1)+2&\overset{?}{<}2\\
-3+2&\overset{?}{<}2\\
-1&\overset{\checkmark}{<}2
\end{align*}

\item $12-4(x-5)$

\Solution
\begin{align*}
12 &\ge -4x+20 && \text{Distributive Property}\\
12-20 & \ge -4x+20-20 && \text{API}\\
-8 & \ge -4x && \text{Simplify}\\
\Bigl(\frac{1}{-4}\Bigr) &\le \Bigl(\frac{1}{-4}\Bigr)(-4x) && \text{MPI}\\
2 &\le x && \text{Multiplicative inverse property}\\
x &\ge 2 && \text{It's easier when the variable comes first.}
\end{align*}
\textbf{Solution set:} $\{x\vert x\ge 2\}$
\begin{figure}[!h]
\centering
\caption{Notice the closed interval on $x=2$. This indicates that 2 is part of the solution set.}
\begin{tikzpicture}[>=stealth']
	\foreach \x in {-5,-4,...,4,5}
		\draw (\x,2pt) -- (\x,-2pt) node [below] {$\x$};
	\draw [<->, thick] (-5.5,0) -- (5.5,0);
	\node (a) at (2,0) [circle, ultra thick, inner sep=0pt, blue!50, text width=5pt, draw,fill=blue] {};
	\draw [->, ultra thick,blue!50!white] (a) -- (5.5,0);
\end{tikzpicture}
\label{chap6fig:3}
\end{figure}

\CHECK{} Test value: $x=3$
\begin{align*}
12&\overset{?}{\ge}-4(x-5)\\
12&\overset{?}{\ge}-4(3-5)\\
12&\overset{?}{\ge}-4(-2)\\
12&\overset{\checkmark}{\ge}8
\end{align*}
	\end{enumerate}
\end{example}
\subsection*{APPLICATION OF LINEAR INEQUALITIES}
The skill of being able to translate sentences into mathematical symbols is a must in successful
problem solving. Some of the \Bold{inequality key words} are the following.
\begin{property}[frametitle={Inequality key words}]
\begin{center}
\begin{tabular}{ll}
at least	 \tikzmark{a1} & \tikzmark{a2} greater than or equal to\\
no more than \tikzmark{b1} & \tikzmark{b2}	less than or equal to\\
more than \tikzmark{c1} & \tikzmark{c2}	greater than\\
at most \tikzmark{d1} & \tikzmark{d2} less than or equal to\\
\end{tabular}
\end{center}
\end{property}
\begin{tikzpicture}[remember picture, overlay,ultra thick, blue,yshift=1pt]
\draw [->] (a1.north) -- (a2.north);
\draw [->] (b1.north) -- (b2.north);
\draw [->] (c1.north) -- (c2.north);
\draw [->] (d1.north) -- (d2.north);
\end{tikzpicture}
\begin{example}
\Item Taxi operators in Metro Manila charges \textpeso40 as flag down rate, in addition to \textpeso3.50 per 300
meters or 2 minutes of waiting time. Macy has no more than \textpeso250 to spend on a ride.
	\begin{itemize}
	\item Write an inequality that represents Macy's situation, assuming that there was no instance that the taxi stopped moving (no waiting time).
	\item How many kilometers can Macy travel without exceeding her limit?
	
	\Solution: Let $k=$ number of \km{} Macy can travel without exceeding her limit
	\begin{equation*}
	\frac{\textpeso 3.50}{300\m}\times \frac{100\m}{1\km}=\textpeso\frac{35}{3}/\km
	\end{equation*}
	\begin{center}
	\begin{tabular}{YYYYY}
	\Tikzmark{e1}{$\dfrac{\mathrm{P}35}{3}$} &  \Tikzmark{e2}{$+$} & \Tikzmark{e3}{40} & \Tikzmark{e4}{$\le$} & \Tikzmark{e5}{250}\\
	& & & & \\
	& & & & \\
	\Tikzmark{f1}{P 35/3 per \km} &  & \Tikzmark{f3}{flag down rate} & \Tikzmark{f4}{no more than} & \Tikzmark{f5}{amount to spend}\\
	\end{tabular}
	\begin{tikzpicture}[remember picture, overlay,ultra thick, blue,yshift=1pt,>=stealth']
\draw [->] (e1.south) -- (f1.north);
\draw [->] (e3.south) -- (f3.north);
\draw [->] (e4.south) -- (f4.north);
\draw [->] (e5.south) -- (f5.north);
\end{tikzpicture}
	\end{center}
	\begin{equation*}
	\frac{35}{3}k\le 250-40\iff \frac{35}{3}k\le 210 \iff 35k\le 630 \iff k\le 18\km
	\end{equation*}
	Thus, Macy cannot go beyond 18 \km.
	\end{itemize}
\Item Salyn has P20,000 in a savings account at the beginning of the year. She wants to have at least
P14,000 in her account by the end of the summer. She withdraws P 500 every week for
miscellaneous expenses.
	\begin{itemize}
	\item Write an inequality that represents Salyn's situation.
	\item How many weeks can Salyn withdraw money from her account?
	
	\Solution{}
	
	Let $w =$ number of weeks Salyn can withdraw money from her savings account
\begin{center}
	\begin{tabular}{YYYY@{}Y@{}}
	\Tikzmark{g1}{20,000} &  \Tikzmark{g2}{$-$} & \Tikzmark{g3}{$500w$} & \Tikzmark{g4}{$\ge$} & \Tikzmark{g5}{14,000}\\
	& & & & \\
	& & & & \\
	& & & & \\
	\Tikzmark{h1}{} & \Tikzmark{h2}{withdraw} & \Tikzmark{h3}{} & \Tikzmark{h4}{at least} & \tikz [remember picture, overlay] \node (h5) [text width=\linewidth, align=center] {desired amount left at the end of summer};\\
	\end{tabular}
	\begin{tikzpicture}[remember picture, overlay,ultra thick, blue,yshift=1pt,>=stealth']
\draw [->] (g2.south) -- (h2.north);
\draw [->] (g4.south) -- (h4.north);
\draw [->] (g5.south) -- (h5.north);
\end{tikzpicture}
	\end{center}
	\begin{align*}
	-500w&\ge 14,000-20,000\\
	-500w&\ge -6,000\\
	\frac{-500w}{-500}&\ge \frac{-6,000}{-500} && \text{division by a negative number}\\
	w&\le 12\; \text{weeks}
	\end{align*}
	Thus, Salyn has a maximum of 12 weeks to withdraw P500 weekly.
	\end{itemize}
\Item Owen wants to order USBs on the Internet. Each USB costs P250.00 and shipping for the entire
order is P200.00. Owen has at most P5,300 to spend.
	\begin{itemize}
	\item Write an inequality that represents Owen's situation.
	\item How many USB's can Owen order with her P5,300?

	\Solution: Let $u =$ number of USBs that Owen can buy
	
	
	\begin{tabular}{YYYYY}
	\Tikzmark{i1}{250u} &  \Tikzmark{i2}{$+$} & \Tikzmark{i3}{200} & \Tikzmark{i4}{$\le$} & \Tikzmark{i5}{5,300}\\
	& & & & \\
	& & & & \\
	& & & & \\
  & & & \Tikzmark{j4}{at most} & \\
	\end{tabular}
	\begin{tikzpicture}[remember picture, overlay,ultra thick, blue,yshift=1pt,>=stealth']
\draw [->] (i4.south) -- (j4.north);
\end{tikzpicture}
	\begin{align*}
	250u&\le 5,100\\
	\frac{250u}{250} & \le \frac{5,100}{250}\\
	\therefore u & \le 20.4
	\end{align*}
	So, Owen can order a maximum of 20 USB's.
	\end{itemize}
\end{example}
\subsection*{Enrichment Lesson: Step-by-step Solving using the Properties of Real Numbers}
Solve $4x-3=25$

\Solution:

\begin{align*}
(4x - 3)+ 3 &= 25 + 3  && \text{Addition Property of Equality}\\
(4x + -3) + 3 &= 25 + 3 && \text{Definition of Subtraction}\\
4x + (-3 + 3) &= 25 + 3 && \text{Commutative Property of Addition}\\
4x + 0 &= 25 + 3 && \text{Inverse Property of Addition}\\
4x &= 25 + 3 && \text{Identity Property of Addition}\\
4x &= 28 && \text{Addition Fact}\\
(4x)\frac{1}{4}&=(29) \frac{1}{4} && \text{Multiplication Property of Equality}\\
(4x)\frac{1}{4}&=7 && \text{Multiplication Fact}\\
\frac{1}{4}(4x)&=7 && \text{Commutative Property of Multiplication}\\
\Bigl( \frac{1}{4}\cdot 4\Bigr)x&=7 && \text{Associative Property of Multiplication}\\
1\cdot x&=7 && \text{Inverse Property of Multiplication}\\
x&=7 && \text{Identity Property of Multiplication}
\end{align*}
\CHECK
\begin{align*}
4 (7) - 3 &\overset{?}{=}25\\
28 - 3 &\overset{?}{=}25\\
25 & \overset{\checkmark}{=}25
\end{align*}
Therefore, $x=7$ is a root or solution of the given equation.
\begin{exercise}
\Item Solve $6(2x-\frac{1}{2})+15=-3(x+1)$ using the procedure above.

\end{exercise}
