\chapter{ALGEBRAIC MANIPULATIONS AND COMMON MISCONCEPTIONS}
\section*{INTRODUCTION}
This module presents common misconceptions of students in using GEMDAS in their algebra
classes, and some mental math and shortcuts in addition, subtraction, multiplication and division.
Also, we review the properties of real numbers as a field, rules on exponents, and hierarchy of
operations. Examples, exercises, and oral assessment are also given at the end of the module.
\section*{OBJECTIVES}
After completing this module, you should be able to:
\begin{enumerate}
\item Recognize properties of real numbers and use them correctly;
\item Recognize properties of exponents and use them correctly;
\item Remember GEMDAS and apply it in simplifying numeric and algebraic expressions;
\item Learn and be aware of the common misconceptions and errors in using GEMDAS and
simplifying numeric and algebraic expressions; and
\item Acquire new tricks and shortcuts in mental math.
\end{enumerate}
\section*{DISCUSSION}
Before we discuss some of the common misconceptions in using GEMDAS, it is just right to
go over with some theoretical background and discussions that covers these misconceptions.
\subsection*{Review on Properties of Real Numbers}
The properties of real numbers can be used to rewrite algebraic expressions. The properties
are true for variables and algebraic expressions as well as for real numbers.

At this point, you are advised to review the properties of real numbers by revisiting Chapter \ref{chap:2}.
\subsection*{Properties of Exponents}
Just as multiplication by a positive integer can be described as repeated addition, \Ital{repeated
multiplication} can be written in what is called \Bold{exponential form}. Let $n$ be a positive integer and let $a$ be a real number. Then the product of $n$ factors of $a$ is given by
\begin{equation}
a^n=\underbrace{a\cdot a\cdot a\cdots a}_{n\, a\text{'s}}\qquad \text{$n$ factors, $a$ is the base and $n$ is the \Bold{exponent}.}
\end{equation}
Let $a$ and $b$ represent real numbers, and let $m$ and $n$ be integers. Then the properties in Table \eqref{chap4tab:1} are always true.
{\renewcommand{\arraystretch}{1.5}
\begin{table}[!h]
\centering
\caption{Properties of Exponents}
\begin{tabular}{>{$\arraybackslash}l<{$\arraybackslash}>{$\arraybackslash}l<{$\arraybackslash}}
\hline
\hline
\mathrm{Property} & \mathrm{Example}\\
a^ma^n=a^{m+n} & x^5(x^4)=x^{5+4}=x^9\\
(ab)^m=a^mb^m & (2x)^3=2^3x^3=8x^3\\
(a^m)^n=a^{mn} & (x^2)^3=x^{2\cdot 3}=x^6\\
\dfrac{a^m}{a^n}=a^{m-n},a\neq0 & \dfrac{x^6}{x^2}=x^{6-2}=x^4,x\neq0\\
\left(\dfrac{a}{b}\right)^m=\dfrac{a^m}{b^m},b\neq0 & \left(\dfrac{x}{2}\right)^3=\dfrac{x^3}{2^3}=\dfrac{x^3}{8}\\
\hline
\end{tabular}
\label{chap4tab:1}
\end{table}}
\subsection*{Historical Note}
Originally, Arabian mathematicians used their words for colors to represent quantities (\LATIN{cosa},
\LATIN{censa},   \LATIN{cubo}). These words were eventually abbreviated to \LATIN{co}, \LATIN{ce}, \LATIN{cu}. \Bold{Rene Descatrtes} (1596-1650)
simplified this even further by introducing the symbols $x$, $x^2$ and $x^3$.
\subsection*{Order of Operations}
The order of operations is a set of rules that tells what sequence to use when simplifying
expressions that contain more than one operation.
\begin{table}[!h]
\centering
\caption{Order of Operations}
\begin{tabular}{ll}
\hline \hline
First: & Perform operations inside grouping symbols.\\
Second: & Simplify powers.\\
Third: & Perform multiplication and division from left to right.\\
Fourth: & Perform addition and subtraction from left to right.\\
\hline
\end{tabular}
\end{table}
Grouping symbols include parentheses $()$, brackets $[]$, braces $\{\}$, fraction bars, radical
symbols, and absolute value symbols.

If an expression contains more than one grouping symbol, simplify the innermost set first.
Within each set, follow the order of operations.

Remember that you have to use the order of operations: \textbf{G}rouping, \textbf{E}xponents,
\textbf{M}ultiplication/\textbf{D}ivision, and \textbf{A}ddition/\textbf{S}ubtraction (\Bold{GEMDAS}).

\subsection*{EXERCISES}
Simplify the following expressions.
\begin{multicols}{2}
\begin{enumerate}
\item $14+(125\times4-125)/5$
\item $(14+3)\times 14-5$
\item $(14+18)\times 9-30/5$
\item $2+18\times(32-40)/2$
\item $(10+8)^2-6$
\item $2+(84\times 9-112)/7$
\item $13+12\times 5-25/5$
\item $12+2\times (216-108)/9$
\item $16+(75\times 8-195)/3$
\item $(7+12)\times 4-56/8$
\end{enumerate}
\end{multicols}
\subsubsection*{Common Misconceptions in Calculation}
Two of the most basic problems in elementary algebra are simplifying numerical expressions
and evaluating algebraic expressions. Simplifying a numerical expression means performing the
indicated operations in proper sequence to obtain a single number. Evaluating an algebraic
expression consists of replacing each of the variables in the expression with given numbers and
simplifying the resulting numerical expression.

The following examples show common exercises that students encounter early in many
algebra textbooks.

\begin{example}
\item Simplify $-2+5(4-6)^2$
\begin{equation*}
\boxed{
\begin{split}
-2+5(4-6)^2\\
			-2+5(-2)^2\\
			-2+5(4)\\
			-2+20\\
			18
\end{split}
}
\end{equation*}
\item Evaluate $5x^2-2xy$ using $x=-2$ and $y=-3$.
\begin{equation*}
\boxed{
\begin{split}
5(-2)^2-2(-2)(-3)\\
5(4)-12\\
20-12\\
8
\end{split}
}
\end{equation*}
\end{example}
% Look for the source of Marquis
Marquis (1988) collected some of the common misconceptions and errors in using GEMDAS
and described the universality of a certain set of errors made by students who are attempting to
transform algebraic expressions. She provided a list of twenty-two such errors in Figure \eqref{chap4fig:1}.
\begin{figure}[!h]
\centering
\caption{Common misconceptions and errors in using GEMDAS}
\begin{multicols}{2}
\begin{enumerate}
\item $|-3|=-3$
\item $3^2\cdot 3^3=9^5$
\item $a^2\cdot a^5=(ab)^7$
\item $x+y-3(z+w)=x+y-3z+w$
\item $\dfrac{r}{4}-\dfrac{6-s}{2}=\dfrac{r-12-2s}{4}$
\item $3a+4b=7ab$
\item $3x^{-1}=\dfrac{1}{3x}$
\item $\sqrt{x^2+y^2}=x+y$
\item $\dfrac{x+y}{x+z}=\dfrac{y}{z}$
\item $\dfrac{1}{x-y}=\dfrac{-1}{x+y}$
\item $\dfrac{x}{y}+\dfrac{r}{s}=\dfrac{x+r}{y+s}$
\item $x\left(\dfrac{a}{b}\right)=\dfrac{ax}{bx}$
\item $\dfrac{xa+xb}{x+xd}=\dfrac{a+b}{d}$
\item $\sqrt{-x}\sqrt{-y}=\sqrt{xy}$
\item If $2(2-z)<12$ then $z<-4$
\item $\cfrac{1}{1-\cfrac{x}{y}}=\dfrac{y}{1-x}$
\item $a^2\cdot a^5=a^{10}$
\item $(3a)^4=3a^4$
\item $\dfrac{a}{b}-\dfrac{b}{a}=\dfrac{a-b}{ab}$
\item $(x+4)^2=x^2+16$
\item $\dfrac{r}{4}-\dfrac{6-s}{4}=\dfrac{r-6-s}{4}$
\item $(a^2)^5=a^7$
\end{enumerate}
\end{multicols}
\label{chap4fig:1}
\end{figure}
Now, let us demonstrate misconceptions in using GEMDAS and incorrect use of algebra in the
following cases, taken from \textcite{larson}, \textcite{common}, \textcite{dawkins2} and \textcite{merlin}.
\paragraph*{Dividing by Zero} Everyone knows that $\dfrac{0}{2}=0$, the problem is that far too many students say that
$\frac{2}{0}=0$ or $\frac{2}{0}=2$. Remember that division by zero is undefined.
\paragraph*{Bad/Lost/Assumed Parenthesis} There are a lot of errors that students commonly make here. The first error is that
students get lazy and decide that parenthesis are not needed at certain steps or tend to
forget about them in the very next step. The other error is that students sometimes don't
understand just how important parentheses really are as often seen in errors made in
exponentiation.
\begin{example}
\item Square $4x$.
\begin{center}
\begin{tabular}{cc}
Correct & Incorrect\\
$(4x)^2=4^2x^2=16x^2$ & $4x^2$\\
\end{tabular}
\end{center}
When dealing with exponents remember that only the quantity immediately to the
left of the exponent gets the exponent. So in the incorrect case, x is the quantity
immediately to the left of the exponent, so we are squaring only x, and 4 is not squared. In
the correct case, the parenthesis is immediately to the left of the exponent so this signifies
that everything inside the parenthesis should be squared.
\Item Square $-3$.
\begin{align*}
(-3)^2&\neq -3^2\\
(-3)(-3)&\neq -(3)(3)\\
9&\neq-9
\end{align*}
Remember that exponent comes before the negative symbol. So, on the right side,
only 3 gets squared before negated.
\Item Convert $\sqrt{7x}$ to fractional exponents.
\begin{center}
\begin{tabular}{cc}
$\sqrt{7x}=(7x)^\frac{1}{2}$ & $\sqrt{7x}=7x^\frac{1}{2}$.
\end{tabular}
\end{center}
\end{example}
\paragraph*{Improper Distribution} Be careful when using distribution property, two errors that are common to
students are as follows.
\begin{example}
\Item Multiply $4(2x^2-10)$.
\begin{center}
\begin{tabular}{cc}
Correct & Incorrect\\
$4(2x^2-10)=8x^2-40$ & $4(2x^2-10)=8x^2-10$\\
\end{tabular}
\end{center}
Make sure you distribute the 4 all the way through the parenthesis. Too often
students just multiply the first term by the 4 and ignore the second term, especially when
the second term is just a number.
\Item Multiply $3(2x-5)^2$.
{\setlength{\abovedisplayskip}{0pt}
\begin{center}
\begin{tabular}[b]{cc}
Correct & Incorrect\\
\parbox[b][]{0.3\linewidth}{$
\begin{aligned}
3(2x-5)^2&=3(4x^2-20x+25)\\
 &=12x^2-60x+75
\end{aligned}
$} &
\parbox[b][]{0.3\linewidth}{
\begin{equation*}
\begin{aligned}
3(2x-5)^2&=(6x-15)^2\\
 &=36x^2-180x+225
\end{aligned}
\end{equation*}
}\\
\end{tabular}
\end{center}
}
Remember that exponentiation must be performed before you distribute any
coefficients through the parenthesis.
\end{example}
\paragraph*{The Mad Slasher} The Mad Slasher is when, in the haste of the moment, start crossing out similar
looking expression on the numerator and denominator. This mistake is understood by
realizing that the operation you are doing when cancelling is division; and when you cancel
the denominator with a single term you are essentially only dividing that single term by the
denominator.

Some bad examples of mad slashing are:
\begin{example}
\Item $\dfrac{4+8}{4}=\dfrac{12}{4}=3$ definitely does not equal $\dfrac{\cancel{4}+8}{\cancel{4}}=1+8=9$.
\Item $\dfrac{x+4}{x}=\dfrac{\cancel{x}+4}{\cancel{x}}=4$.
\Item $\dfrac{(x-3)x-x^2-x^2(x+1)}{x-3}=\dfrac{\cancel{(x-3)}x-x^2(x+1)}{\cancel{x-3}}=x-x^2(x+1)$
\end{example}
\paragraph*{Linear Dysfunction Disorder (LDD)}
Much like in the Mad Slasher case, an instinctual desire to simplify as much as
possible can lead the students to a case of LDD, which is an incorrect use of linearity
properties. The following are some examples of LDD.
{\setlength{\abovedisplayskip}{-12pt}
\begin{center}
\begin{tabular}[t]{cc}
Incorrect & Correct\\
$(x+9)^2=x^2+9^2=x^2+81$ &
\parbox[t][]{0.3\textwidth}{
\begin{align*}
(x+9)^2&=(x+9)(x+9)\\
			&=x^2+9x+9x+81\\
			&=x^2+18x+81
\end{align*}
}\\
$\sqrt{x^2+64}=\sqrt{x^2}+\sqrt{64}=x+8$ & $\sqrt{x^2+64}=$ (cannot be simplified further)\\
$\dfrac{x^2}{x^2-4}=\dfrac{x^2}{x^2}-\dfrac{x^2}{4}=1-\dfrac{x^2}{4}$ & $\dfrac{x^2}{x^2-4}=\dfrac{x^2}{(x+2)(x-2)}=$ (cannot be simplified further)\\
$2^{x+4}=2^x+2^4$ & $2^{x+4}=2^x+2^4$ (by the law of exponents)\\
\end{tabular}
\end{center}}

\subsection*{Mental Math}
\begin{quote}
\textit{Figuring out answers in our heads is an important skill. Give it a starting role in your math
teaching. - Marilyn Burns}
\end{quote}

Every day in the advanced world, we face situations that call for adding, subtracting,
multiplying, or dividing. We figure tips in restaurant, decide when to leave home to get to the
movies on time, estimate the price of sale item, keep track of what we are spending while shopping
in the supermarket, double and halve recipes, and so on. Figuring in our own head is such an
important life skill that is why mental math is, in one way or another, a very important skill that we
and our students should learn.

We present some mental math tricks and shortcuts found in \textcite{benjamin}, \textcite{short}, and \textcite{stephens}.

\subsubsection*{Mental Math Tricks}
In this part, we list down some mental math tricks and shortcuts.

\subsubsubsection{Addition}
\paragraph*{Adding left to right}
\begin{example}
\Item $326 + 678 + 245 + 567 =$

Add the hundreds digit first from left to right, then add the tens digits, then the units digits.

900, 1100, 1600, 1620, 1690, 1730, 1790, 1796, 1804, 1809, then 1816

\Item $1757 + 5783 =$

6000, 6700, 7400, 7450, 7530, 7537, then 7540

Note: Look for opportunities to combine numbers to form 10, 100, 1000 and etc. between
numbers that are not necessarily next to each other. Practice!
\end{example}
\paragraph*{Distributive property}
\begin{example}
\Item $5 \times 17 + 5 \times 3 =$
Note that using Distributive Property of Multiplication over Addition:

$5 \times 17 + 5 \times 3 = 5 \times (17 + 3) = 5 \times 20 = 100$
\Item $6\times 78$

Note that using Distributive Property of Multiplication over Addition:

$6 \times 78 = 6 \times (70 + 8) = (6 \times 70) + (6 \times 8) = 420 + 48 = 468$

\Item $7\times 99=$

Note that using Distributive Property of Multiplication over Addition:
\begin{equation*}
7 \times 99 = 7 \times (100 - 1) = (7 \times 100) - (7 \times 1) = 700 - 7 = 693
\end{equation*}
\end{example}

\subsubsubsection{Subtraction}
\paragraph*{Round the Subtrahend}
\begin{example}
\Item $496 - 279 =$

Add a number to the subtrahend to round up to the nearest multiple of 10, then add same
number to the minuend and subtract.
\begin{equation*}
496 - 279 = (496 + 4) - (279 + 4) = 500 - 283 = 217
\end{equation*}
\end{example}
\paragraph*{Round the Minuend}
\begin{example}
\Item $496-279 =$

Add a number to the minuend to round up to the nearest multiple of 10, then add same
number to the subtrahend and subtract.
\begin{equation*}
496 - 279 = (496 + 21) - (279 + 21) = 517 - 300 = 217
\end{equation*}
\end{example}
\subsubsubsection{Multiplication and Squaring}
\paragraph*{Multiply by 50, 25 or 75}
\begin{example}
\Item $24\times 50$

Multiply by 100 and divide by 2 (or vice-versa).
\begin{equation*}
24 \times 50 = 24 \times 100 \div 2 = 2400 \div 2 = 1200
\end{equation*}
\Item $96\times 25=$

Multiply by 100 and divide by 4 (or vice-versa).
\begin{equation*}
96 \times 25 = 96 \times 100 \div 4 = 9600 \div 4 = 2400
\end{equation*}
\Item $56\times 75$

Multiply by 100 and divide by 4, then multiply by 3.
\begin{equation*}
56 \times 75 = 56 \times 100 \div 4 \times 3 = 5600 \div 4 \times 3 = 1400 \times 3 = 4200
\end{equation*}
\end{example}
\paragraph*{Squaring}
\begin{example}
\Item $61^2=$

Using the special product: $(a+b)^2=a^2+2ab+b^2$
\begin{equation*}
61^2 = (60 + 1)^2 = 60^2 + 2(60)(1) + 1^2 = 3600 + 120 + 1 = 3721
\end{equation*}
\Item $78^2=$

Using the special product: $(a-b)^2=a^2-2ab+b^2$
\begin{equation*}
78^2 = (80 - 2)^2 = 802 - 2(80)(2) + 2^2 = 6400 - 320 + 4 = 6080 + 4 = 6084
\end{equation*}
\end{example}
\paragraph*{Multiplying two numbers using difference of two squares}
\begin{example}
\Item $21\times 19=$

Using the special product: $a^2-b^2=(a+b)(a-b)$
\begin{equation*}
21 \times 19 = (20 + 1)\times (20 - 1) = 202 - 12 = 400 - 1 = 399
\end{equation*}
\Item $48\times 52=$

Using the special product: $(a-b)^2=a^2-2ab+b^2$
\begin{equation*}
48 \times 52 = (50 - 2)\times (50 + 2) = 502 - 22 = 2500 - 4 = 2496
\end{equation*}
\end{example}
\subsubsubsection{Division}
\paragraph*{Divide by 5, 25 and 50}
\begin{example}
\Item $365\div 5=$

Multiply by 2 and divide by 10.
\begin{equation*}
365 \div 5 = 365 \times 2 \div 10 = 730 \div 10 = 73
\end{equation*}
\Item $234\div 50=$

Multiply by 2 and divide by 100.
\begin{equation*}
234 \div 50 = 234 \times 2 \div 100 = 468 \div 100 = 4.68
\end{equation*}

\Item $212\div 25=$

Multiply by 4 and divide by 100.
\begin{equation*}
212 \div 25 = 212 \times 4 \div 100 = 848 \div 100 = 8.48
\end{equation*}
\end{example}
\paragraph*{Divide by the factors of the divisors one at a time}
\begin{example}
\Item $728\div 14=$

The factors of 14 are 2 and 7.
\begin{equation*}
728 \div 14 = (728 \div 7) \div 2 = 104 \div 2 = 52
\end{equation*}
\Item $1344\div 24=$

The factors of 24 are 6 and 4.
\begin{equation*}
1344 \div 24 = (1344 \div 6) \div 4 = 224 \div 4 = 56
\end{equation*}
\end{example}
\section*{Suggested Activities}
\subsection*{Activity 1: Common Misconceptions}
Correct the mathematical statement and describe the error.

Potential Error \hfil Correct Form \hfil Comment
\begin{enumerate}
\item $a-(x-b)=a-x-b$
\item $(a+b)^2=a^2+b^2$
\item $\left(\dfrac{1}{2}a\right)\left(\dfrac{1}{2}b\right)=\dfrac{1}{2}(ab)$
\item $(3x+6)^2=3(x+2)^2$
\item $\dfrac{a}{x+b}=\dfrac{a}{x}+\dfrac{a}{b}$
\item $\left(\dfrac{1}{3}\right)x=\dfrac{1}{3x}$
\item $\left(\dfrac{1}{x}\right)+2=\dfrac{1}{x+2}$
\item $(x^2)^3=x^5$
\item $2x^3=(2x)^3$
\item $\dfrac{1}{x^2-x^3}=x^{-2}-x^{-3}$
\item $\sqrt{x^2+a^2}=x+a$
\item $\dfrac{a+bx}{a}=1+bx$
\item $\dfrac{a}{a}=1$
\item $\dfrac{a+ax}{a}=a+x$
\item $\sqrt{-x+a}=-\sqrt{x-a}$
\item $\dfrac{1}{a^{-1}+b^{-1}}=\left(\dfrac{1}{a+b}\right)^{-1}$
\item $x(2x-1)^2=(2x^2-x)^2$
\item $\dfrac{2x^2+1}{5x}=\dfrac{2x+1}{5}$
\item $\dfrac{3}{x}+\dfrac{4}{y}=\dfrac{7}{x+y}$
\item $a\left(\dfrac{x}{y}\right)=\dfrac{ax}{ay}$
\end{enumerate}
\subsection*{Activity 2: Calculator Debate}
Divide the participants into 6 groups. Give a copy of the picture in Figure \eqref{chap4fig:2} and make each
group discuss why the 2 calculators (same brand at that!) have different
answers. Determine if one of the calculators is correct. Facilitate a debate and
resolve the issue.

\begin{figure}[!h]
\centering
\caption{Which is correct?}
\includegraphics[width=0.8\textwidth]{casio}
\label{chap4fig:2}
\end{figure}
\subsection*{Activity 3: Oral Quiz Bee}
\subsubsection*{Materials}
Flashcard, Marker, Masking tape, Improvised buzzer
\subsubsection*{Preparation for the Quiz Bee}
The teacher-in-charge will prepare flashcards where the questions or mathematical statements will
be written. Number of the corresponding question or mathematical statement will be put on in front
part of the flashcard.
\begin{figure}[!h]
\centering
\caption{Sample Flash Flash Card}
\subfigure[Front of flash card]{\label{chap4fig:3}
\begin{tikzpicture}
\node [rectangle,draw,text width=2in,minimum height=3in,align=center] {
\Huge \bfseries 1
};
\end{tikzpicture}
}
\qquad
\subfigure[Back of flash card]{\label{chap4fig:4}
\begin{tikzpicture}
\node [rectangle,draw,text width=2in,minimum height=3in,align=center] {\Large 
$7\times 36+7\times 4$
};
\end{tikzpicture}
}%
\end{figure}
And then the flashcards will be pasted on the board as in Table \eqref{chap4tab:2}
\begin{table}[!h]
\centering
\caption{Placement of flashcards}
\begin{tabular}{|l|l|l|l|l|}
\hline
1 & 2 & 3 & 4 & 5\\ \hline
6 & 7 & 8 & 9 & 10\\ \hline
11 & 12 & 13 & 14 & 15\\ \hline
16 & 17 & 18 & 19 & 20\\ \hline
21 & 22 & 23 & 24 & 25\\ \hline
\end{tabular}
\end{table}
First row contains easy question, second and third rows contain average questions, while fourth and
fifth rows contain difficult questions.
\subsubsection*{Mechanics}
\begin{enumerate}
\item Students will be divided into 5 groups. Each group will have a team captain who will serve as
the leader of the group.
\item Each group will be provided with an improvised buzzer.
\item The questions to be asked are shown with corresponding points and time limits.
\begin{center}
\begin{tabular}{|l|l|l|}
\hline
Easy & 2 point & 15 seconds\\ \hline
Average & 3 points & 20 seconds\\ \hline
Difficult & 5 points & 30 seconds\\ \hline
\end{tabular}
\end{center}
\item No writing of computations, just solve mentally.
\item The teacher-in-charge will choose the first question to be asked. Then the first group who
buzzed up will be given the opportunity to answer the question. If the first group did not
answer the question correctly, other groups will be given the chance to answer.
\item The group who answered the question correctly will have the right to choose the number of
the next question to be asked.
\item Scores will be tallied accordingly.
\end{enumerate}
\subsubsection*{Set of Questions}
\begin{center}
\begin{tabular}{|l|>{$}l<{$}|l|l|l|}
\hline
\# & \parbox[t][]{0.25\linewidth}{\centering Question/Math Expression} & \multicolumn{3}{c|}{WHAT TO DO?}\\ \cline{3-5}
 & & \parbox[t][]{0.2\linewidth}{\centering Simplify/Solve} & \parbox[t][]{0.2\linewidth}{\centering Correct and Describe the Error} & \parbox[t][]{0.15\linewidth}{\centering TRUE or FALSE. Check for divisibility}\\ \hline
1 & 7\times 36+7\times 4 & * & & \\ \hline
2 & (14+3)\times 14-5 & * & & \\ \hline
3 & (a+b)^2=a^2+b^2 & & * & \\ \hline
4 & 174+268+275+547 & * & & \\ \hline
5 & 67348 \div 4 & & & *\\ \hline
6 & (7+12)\times4-56/8 & * & & \\ \hline
7 & 2+18\times (32-40)/2 & * & & \\ \hline
8 & \dfrac{a+bx}{a}=1+bx & & * & \\ \hline
9 & \dfrac{a}{a}=1,\,\text{for all $a$} & & * & \\ \hline
10 & 1^2+1^2+2^2+3^2+5^2+8^2 & * & & \\ \hline
11 & 28\div 11+82\div 11 & * & & \\ \hline
12 & 4425575 \div 9 & & & *\\ \hline
13 & 7.5\times 16 & * & & \\ \hline
14 & \dfrac{1}{x^2-x^3}=x^{-2}-x^{-3} & & * & \\ \hline
15 & a\left(\dfrac{x}{y}\right)=\dfrac{ax}{ay} & & * & \\ \hline
16 & \sqrt{7921} & * & & \\ \hline
17 & 53867 \div 11 & & & * \\ \hline
18 & \dfrac{2x^2+1}{5x}=\dfrac{2x+1}{5} & & * & \\ \hline
19 & 73488 \div 8 & & & * \\ \hline
20 & \sqrt{21316} & * & & \\ \hline
21 & \dfrac{1}{a^{-1}+b^{-1}}=\left(\dfrac{1}{a+b}\right)^{-1} & & * & \\ \hline
22 & 12+2\times(216-108)/9 & * & & \\ \hline
23 & 998^2 & * & & \\ \hline
24 & \text{\parbox[t][]{0.3\linewidth}{If $a=3$ and $b=5$, then $(a-b)(a^2+ab+b^2)$}} & * & & \\ \hline
25 & 13855 \div 17 & & & * \\ \hline
\end{tabular}
\end{center}
\subsection*{Activity 4: Proving Misconception}
The following is the proof that $1 = 2$. What is wrong with the proof?
\begin{center}
\begin{tabular}{r>{$}l<{$}l}
1. & a=b & We'll start assuming this to be true.\\
2. & ab=b^2 & Multiply both sides by $a$.\\
3. & ab-b^2=a^2-b^2 & Subtract $b^2$ from both sides.\\
4. & b(a-b)=(a+b)(a-b) & Factor both sides.\\
5. & b=a+b & Divide both sides by $a-b$ \hphantom{phantom}\\
6. & b=2b & Recall we started off by assuming $a=b$.\\
7. & 1=2 & Divide both sides by $b$.\\
\end{tabular}
\end{center}
\subsection*{Activity 5: Mental Challenge}\label{chap4:sec1}
Answer the following mentally.
\begin{multicols}{2}
\begin{enumerate}
\item $174 + 268 + 275 + 547$
\item $7 \times 36 + 7 \times 4$
\item $39^2$
\item $46^2$
\item $37 \times 43$
\item $77^2$
\item $7.5 \times 16$
\item $876 - 289$
\item $921 - 388$
\item $3912 \div 12$
\end{enumerate}
\end{multicols}
\begin{tikzpicture}
\node [rotate=180]{
Answers.
\begin{inparaenum}
\item 1264
\item 280
\item 1521
\item 2116
\item 1591\\
\item 5929
\item 120
\item 587
\item 533
\item 326
\end{inparaenum}
};
\end{tikzpicture}
