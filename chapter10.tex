\chapter{INTRODUCTION TO STATISTICS}
\section*{INTRODUCTION}
This module present a discussion on statistics to include key concepts, uses and
importance of statistics and probability, data collection and the different forms of data
representation. It will also discuss the measures of central tendency for ungrouped and grouped
data. Every lesson will present examples, application and sample assessment tools. You will
appreciate more the importance of statistics which you will also share to your students.
\section*{OBJECTIVES}
After completing this module, you should be able to:
\begin{enumerate}
\item Explain the basic concepts, uses, and the branches of statistics;
\item Collect or gather statistical data and organize in a frequency table according to some
systematic considerations;
\item Use appropriate graph to represent organized data: pie, bar graph, line graph, and
histogram;
\item Analyze, interpret accurately and draw conclusions from graphic and tabular
presentations of statistical data;
\item Find the mean, median and mode of statistical data; and
\item Describe the data using information from the mean, median and mode.
\end{enumerate}
\section*{DISCUSSION}
\begin{definition}[Statistics]
\Bold{Statistics} is the science of conducting studies to collect, organize, summarize,
analyze and draw conclusions from data (Bluman, 2004). It is a science of
collection and classification of facts in the basis of relative number or
occurrence.
\end{definition}
The \Bold{collection} or \Bold{gathering of data} may be undertaken using interview, questionnaires,
tests, observation, registration, and experiments. \Bold{Organization of data} is the manner of
presenting the gathered data into tabular, textual or graphical form. The \Bold{summarized data} will
be subjected to \Bold{analysis} in order to extract relevant data and information using statistical tools
or techniques. \Bold{Interpretation of data} refers to the drawing of conclusions or inferences from
the analyzed data (Basilio, Faith B., Edna A. Chua, Maria T. Juwanan, 2003).
\subsection*{Uses of Statistics}
Statistics has many uses. Among the contributions of statistics are the following:
\begin{itemize}
\item Statistics provides us with the ways and means of expressing our thoughts in the most
definite and exact way feasible. Example: There are more females in the first year than
males; it will rain on June.
\item Statistics provides us with numbers and figures with which to describe completely and
   accurately the characteristics of certain phenomena. Example: There are 25 females in
  the first year and 20 are males; it will rain on June 10.
\item Statistics enables us to express the results of research activities meaningfully and
   present them in a form easily understood and read. Example: in tables or graphs.
\item Statistics allows us to draw inferences or conclusions from characteristics of given data.
\end{itemize}
